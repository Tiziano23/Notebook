\documentclass{article}
\usepackage[utf8]{inputenc}
\usepackage[italian]{babel}
\usepackage[margin=1.25in]{geometry}
\usepackage{mathptmx}

\usepackage[dvipsnames]{xcolor}
\usepackage{listings}
\usepackage[outputdir=output]{minted}

\usepackage{hyperref}
\usepackage{graphicx}

\usepackage{amsfonts, amssymb, amsmath, amsthm}
\usepackage{stmaryrd}
\usepackage[thicklines]{cancel}

\hypersetup{
  colorlinks=true,
  linkcolor=black,
  filecolor=magenta,
  urlcolor=blue,
}

\definecolor{codekeyword}{HTML}{7cd2e2}
\definecolor{codestring}{HTML}{7df0d2}
\definecolor{codenumber}{HTML}{000000}
\definecolor{codecomment}{HTML}{7df0a1}
\definecolor{codebackground}{HTML}{1E1E1E}
\lstdefinelanguage{pseudolang}
{
  keywords={let, if, else, for, while, switch, case, return},
  string=[b]",
  comment=[l]//
}
\lstdefinestyle{pseudostyle}
{
  keywordstyle=\color{codekeyword},
  stringstyle=\color{codestring},
  numberstyle=\tiny\color{codenumber},
  commentstyle=\color{codecomment},
  backgroundcolor=\color{codebackground},
  basicstyle=\ttfamily\footnotesize\bfseries\color{White},
  breakatwhitespace=false,
  breaklines=true,
  captionpos=t,
  keepspaces=true,
  numbers=left,
  numbersep=5pt,
  showspaces=false,
  showstringspaces=false,
  showtabs=false,
  tabsize=2
}
\lstset{language=pseudolang, style=pseudostyle}

\newtheorem*{thm}{Teorema}
\theoremstyle{definition}
\newtheorem*{dfn}{Definizione}
\newtheorem*{oss}{Osservazione}
\newtheorem*{dms}{Dimostrazione}
\newtheorem*{eg}{Esempio}
\newtheorem*{nb}{N.B}

\newcommand{\quoted}[1]{``#1''}
\newcommand{\angled}[1]{\langle #1 \rangle}
\newcommand{\verted}[1]{\left\lvert #1 \right\rvert}
\newcommand{\bbracket}[1]{\llbracket #1 \rrbracket}
\newcommand{\numN}{\mathbb{N}}
\newcommand{\numZ}{\mathbb{Z}}
\newcommand{\numQ}{\mathbb{Q}}
\newcommand{\numR}{\mathbb{R}}

\newcommand{\calA}{\mathcal{A}}
\newcommand{\calB}{\mathcal{B}}
\newcommand{\calC}{\mathcal{C}}
\newcommand{\calD}{\mathcal{D}}
\newcommand{\calE}{\mathcal{E}}
\newcommand{\calF}{\mathcal{F}}
\newcommand{\calG}{\mathcal{G}}
\newcommand{\calH}{\mathcal{H}}
\newcommand{\calI}{\mathcal{I}}
\newcommand{\calJ}{\mathcal{J}}
\newcommand{\calK}{\mathcal{K}}
\newcommand{\calL}{\mathcal{L}}
\newcommand{\calM}{\mathcal{M}}
\newcommand{\calN}{\mathcal{N}}
\newcommand{\calO}{\mathcal{O}}
\newcommand{\calP}{\mathcal{P}}
\newcommand{\calV}{\mathcal{V}}

\newcommand{\FTerm}{\mathcal{F}Term}

\newcommand{\gap}{\hspace{.4cm}}
\newcommand{\ggap}{\hspace{.6cm}}
\newcommand{\Gap}{\hspace{1cm}}
\newcommand{\GAP}{\hspace{2.5cm}}

\newcommand{\funcs}[3]{#1: #2 \rightarrow #3}
\newcommand{\ddfrac}[2]{\frac{\displaystyle #1}{\displaystyle #2}}

\graphicspath{ {./images/} }

\title{
    Programmazione Algoritmica \\
    \large Corso di problem solving \& computational thinking
}
\author{Tiziano Marzocchella}
\date{A.A. 2022/2023}

\begin{document}
\input{../include/first_page}

\section{Informazioni sul corso}
L'obiettivo di questo corso è quello di imparare a risolvere problemi logici, acquisendo la capacità fondamentale di astrazione dei problemi.
\subsection*{Esame}
\begin{enumerate}
  \item 5 prove in itinere (compiti in classe), ognuna con un punteggio massimo di 8pt, direttamente riportati al punteggio finale.\\
        Vengono prese in considerazione solo le migliori 4.\\
        Una prova è considerata \textbf{superata} se il punteggio totalizzato è di almeno 4pt.\\
        Se la somma dei punteggi delle suddette prove in itinere supera 18pt si viene direttamente ammessi all'esame orale.
  \item Esame orale
\end{enumerate}
\subsection*{Materiale}
\begin{itemize}
  \item Sulla \href{https://evo.di.unipi.it/student/courses/7/practices}{piattaforma di esercitazioni online} è possibile trovare esercizi di preparazione.
  \item \href{https://classroom.google.com/c/NDg5NzMxMzU4ODAx?cjc=qwfm6jd}{Google classroom}
  \item \href{https://t.me/c/1708877199/46}{Libro}
  \item \href{https://docs.google.com/document/d/1XU_2rwPUkni_s4h8Q-bLkLF9nwT4dSQA4DjgBqMaulg/edit?usp=drive_web&authuser=0}{Checklist}
\end{itemize}

\pagebreak

\section{Introduzione}
\subsection*{Problem solving}
Quando si parla di problem solving, si intende l'insieme di capacità che ci permettono di analizzare una situazione problematica e trovare una soluzione. Per trovare soluzioni sono ottime le capacità di \textbf{astrazione} e \textbf{decomposizione}. Fare astrazione di un problema significa trasformare la sua descrizione in linguaggio naturale, quindi ad \quoted{altissimo} livello, in una descrizione sistematica, nel nostro caso \quoted{comprensibile} da una macchina. Noi ci fermeremo al codice sorgente, ma il compilatore scende fino al più basso livello dei linguaggi informatici, il linguaggio macchina. La \textbf{decomposizione} invece consiste nella suddivisione di un problema in sotto problemi, rendendo più facile il processo di soluzione del problema iniziale. Questo perché si può pensare alla soluzione di ogni sotto problema individualmente e unire tutte le \quoted{sotto-soluzioni} in un secondo momento. Queste due tecniche sono ampiamente utilizzate nella realizzazione di \textbf{algoritmi}, ovvero un metodo per processare e trasformare informazioni.\\
Un algoritmo è una sequenza finita di azioni da eseguire, esprimibili con una quantità finita di tempo e spazio, che trasforma uno o più valori come \emph{input} in uno o più valori come \emph{output}.
\subsection*{Ciclo di vita di un sistema software}
Un qualsiasi sistema software ha un ciclo di vita formato da quattro fasi, ovvero \emph{specifica}, \emph{soluzione}, \emph{codifica}, \emph{esecuzione}.
\subsubsection*{Specifica}
La fase di specifica serve a definire e capire il problema da risolvere.
Dare una \emph{specifica} di un problema significa definirlo in maniera più o meno formale, considerando ogni possibile caso.
In generale, bisogna indicare lo stato iniziale del programma (\textbf{input}) e cosa ci si aspetta come risultato (\textbf{output}).
\subsubsection*{Soluzione}
Per trovare una soluzione al problema, applichiamo le tecniche di \emph{astrazione} e \emph{decomposizione}.
Dobbiamo quindi progettare una sequenza di azioni, ovvero tradurre il nostro ragionamento in un algoritmo. Può quindi tornare utile \underline{crearsi uno schema mentale della soluzione del problema}.
Un algoritmo va progettato pensando in alto livello. Poi, solo dopo aver fissato la soluzione più ottimale possibile, si può passare alla codifica.
Un algoritmo ha le seguenti proprietà:
\begin{itemize}
  \item Finitezza, sequenza di passi finita
  \item Non ambiguità, ogni singolo passo deve essere \emph{non ambiguo}
  \item Eseguibilità, l'algoritmo deve poter essere eseguito da una macchina.
\end{itemize}
%Immagine ciclo di vita software
\pagebreak

\section{Linguaggi di programmazione (LdP)}
\subsection{Nozioni di base}
Un linguaggio di programmazione è uno strumento che utilizziamo per scrivere codice, che nel nostro caso sarà un codice più astratto del classico programma \emph{C} o \emph{Java}.
\subsection*{Paradigmi di programmazione}
I linguaggi di programmazione possono essere suddivisi in due grandi famiglie, quella dei linguaggi \emph{imperativi} (ad es. Java, C, JavaScript) oppure quella dei linguaggi \emph{dichiarativi}. Le diverse tipologie di linguaggi contenuti in queste famiglie rappresentano diversi paradigmi di programmazione.
Al giorno d'oggi però quasi tutti i moderni linguaggi di programmazione sono \textbf{multi-paradigma}, ovvero fanno uso di più paradigmi contemporaneamente. Noi vedremo linguaggi \emph{procedurali}, \emph{funzionali}, e \emph{orientati ad oggetti}
\subsection*{Caratteristiche di un linguaggio di programmazione}
Ogni linguaggio di programmazione ha le due seguenti caratteristiche fondamentali:
\begin{itemize}
  \item \textbf{Sintassi}, ovvero un insieme di regole che definiscono ciò che viene considerato codice corretto o \emph{ben formato}, e che sono divise in due tipi:
        \begin{itemize}
          \item \textbf{Grammatica} \(\rightarrow\) è utile per definire come mettere insieme le parole.\\
                E.g. schema: \textless soggetto\textgreater\textless verbo\textgreater\textless complemento oggetto\textgreater
          \item \textbf{Lessico} \(\rightarrow\) è l'insieme delle parole che possono essere costruite a partire da un insieme di simboli atomici. Le parole \emph{legali} sono il sottoinsieme del lessico che consideriamo accettate (E.g. il dizionario per la lingua italiana).
        \end{itemize}
  \item \textbf{Semantica}, ovvero ciò che definisce se il codice scritto ha un significato.
\end{itemize}
\subsection*{Compilazione del codice}
Per passare da un linguaggio di programmazione, di un qualsiasi livello, ad un linguaggio \quoted{comprensibile} ed eseguibile dalla macchina, utilizziamo il \emph{compilatore} o \emph{interprete}. Il \emph{compilatore} \quoted{traduce} \textbf{l'intero sorgente} in un eseguibile, mentre l'\emph{interprete} \quoted{traduce} e esegue il sorgente \textbf{istruzione per istruzione}. Questi due strumenti hanno un funzionamento ben definito, organizzato in diverse fasi. Queste fasi sono:
\begin{itemize}
  \item Analisi lessicale (\textbf{scanner}), controlla che tutti i termini utilizzati siano previste dal LdP utilizzato.
  \item Analisi sintattica (\textbf{parser}), controlla che le parole siano combinate in modo corretto.
  \item Analisi semantica (\textbf{type check}), controllo sui tipi (?)
  \item Generazione codice oggetto, genera l'eseguibile
  \item Linking
\end{itemize}
\subsection*{Esecuzione del codice}
\begin{itemize}
  \item Architettura di Von-Neumann
  \item Ciclo Fetch-Execute
\end{itemize}
\pagebreak

\subsection{Definizione formale di linguaggio}
In questa sezione daremo una definizione formale di linguaggio di programmazione, partendo dalle basi. Alla base di questa definizione troviamo gli \emph{alfabeti}, ovvero un insieme (matematico) composto da caratteri. I caratteri rappresentano la nostra unità di informazione nell'alfabeto. \\
Avremo quindi bisogno di:
\begin{itemize}
  \item Definire un \emph{alfabeto}
  \item Definire l'operazione di \emph{concatenazione} fra caratteri dell'alfabeto
  \item Definire una \emph{stringa}
\end{itemize}
\subsection*{Alfabeto}
Un alfabeto è un insieme \textbf{finito} e \textbf{non vuoto} di \emph{simboli} (caratteri).
\subsection*{Stringa}
Dati due caratteri \(a, b \in A\), definiamo la concatenazione tra di essi come \(a \cdot b = ab\). \\
Il risultato della concatenazione di caratteri, ovvero elementi di un alfabeto, è detto \textbf{stringa}.
\subsection*{Proprietà di una stringa}
\begin{itemize}
  \item Lunghezza della stringa: \(\lvert s \rvert = \text{ \# di simboli nella stringa}\)
  \item Concatenazione fra due stringhe \(x\) e \(y\): \(x \cdot y = \text{tutti i simboli di \(x\) seguiti da tutti i simboli di \(y\)}\)
  \item Esponenziale di una stringa
        \begin{align*}
          x^0    & = \epsilon                       & x^0 & = \epsilon \\
          x^1    & = x \cdot x^0 = x \cdot \epsilon & x^1 & = ab       \\
          \vdots &                                  & x^2 & = abab     \\
          x^n    & = x \cdot x^{n-1}                & x^3 & = ababab
        \end{align*}
  \item Prefisso di una stringa: \(pre(n,s) = \text{i primi \(n\) caratteri della stringa \(s\)}\)
  \item Suffisso di una stringa: \(suf(n,s) = \text{gli ultimi \(n\) caratteri della stringa \(s\)}\)
        %\item Sotto stringa: \[ substr(n1,n2,s) = pre(n1,s) \cdot suf(n2,s) \]
\end{itemize}
\subsection*{Chiusura di Kleene}
Definiamo l'insieme di tutte le stringhe generabili a partire da un certo alfabeto \(A\) come l'insieme che contiene tutte le possibili stringhe di qualsiasi lunghezza che si possono formare concatenando i simboli dell'alfabeto a cui viene applicata. \\
Ad esempio:
\begin{center}
  Se \(A = \{ 0, 1 \}\) allora \(A^* = \) tutti i numeri binari.
\end{center}
Esiste anche una variante della chiusura di Kleene, detta chiusura positiva, perché esclude la stringa vuota \(\epsilon\), definita come l'insieme di tutte le stringhe di \(A^*\) di lunghezza \(\geq 1\).
\subsection*{Linguaggio}
Diamo quindi una definizione formale e matematica di linguaggio, ovvero l'insieme di stringhe \emph{legali} formate a partire da un alfabeto secondo le regole di una \textbf{grammatica}.
\subsection*{Osservazioni}
\begin{itemize}
  \item Il linguaggio vuoto è un insieme che non contiene alcun elemento. \(\varnothing = \) linguaggio vuoto.
  \item Il linguaggio \(\{ \epsilon \}\) è il linguaggio contenente solo la stringa vuota.
  \item Un generico linguaggio \(L\) è sottoinsieme della chiusura di Kleene su un alfabeto \(A\). \(L \subseteq A^*\)
\end{itemize}
\pagebreak

\subsection{Sintassi}
\subsection*{Nozioni di base}
Generalmente i linguaggi moderni sono sottoinsiemi di ASCII* e i programmi di tali linguaggi non sono altro che stringhe di caratteri ASCII. I LdP però non sono l'insieme completo ASCII*, ma solo un sotto insieme, che viene individuato secondo delle specifiche regole di sintassi.\\
Per definire quali stringhe appartengono al nostro linguaggio \(L\), potremmo enumerarle una ad una, ma questo processo sarebbe impraticabile per insiemi infiniti, come il nostro \(L\). Quindi, si può procedere in 3 modi diversi:
\begin{itemize}
  \item Seguendo il metodo \emph{generativo}, possiamo definire una \emph{grammatica} in grado di poter generare l'insieme di stringhe legali
  \item Seguendo il metodo \emph{riconoscitivo}, potremmo creare un \emph{automa} in grado di riconoscere le stringhe legali
  \item Seguendo il metodo \emph{algebrico}, potremmo definire l'insieme delle stringhe legali a partire dalle soluzioni di un sistema di equazioni algebriche
\end{itemize}
Noi utilizzeremo il metodo \textbf{generativo}.
\subsection*{Grammatica}
Una quadrupla \( G = \left\langle N, \Sigma, P, S \right\rangle\) dove:
\begin{itemize}
  \item \( N \neq \varnothing \) è un insieme di simboli \emph{non terminali}
  \item \( \Sigma \) è un alfabeto di simboli \emph{terminali}
  \item \( P \subseteq (N \cup \Sigma)^+ \times (N \cup \Sigma)^* \) è l'insieme delle produzioni, dove ogni produzione è una coppia
  \item \( S \in N \) è il simbolo iniziale (starting symbol)
\end{itemize}
Esistono alcune convenzioni sulla scrittura delle grammatiche:
\begin{itemize}
  \item \(A, B, \ldots \in N\)
  \item \(a, b, \ldots \in \Sigma\)
  \item \(\alpha, \beta, \ldots \in (N \cup \Sigma)^*\)
  \item \(x, y, \ldots \in \Sigma^*\)
  \item Per le produzioni: \((\alpha, \beta) \text{ oppure } \alpha ::= \beta\)
\end{itemize}
\subsubsection*{Esempio di grammatica: Grammatica delle espressioni aritmetiche}
Considerati:
\begin{itemize}
  \item gli operatori binari: + e *
  \item le parentesi ()
  \item i numeri naturali (operandi)
\end{itemize}
Possiamo definire la grammatica delle espressioni aritmetiche con le seguenti produzioni:
\begin{enumerate}
  \item \(\left( E, N \right)\)
  \item \(\left( E, \left(E\right) \right)\)
  \item \(\left( E, E + E \right)\)
  \item \(\left( E, E * E \right)\)
  \item \(\left( N, 0 \right)\)
  \item \(\left( N, 1 \right)\)
  \item \(\vdots\)
  \item \(\left( N, 9 \right)\)
\end{enumerate}
La notazione \(\left(A, B\right)\) è una produzione che indica la sostituzione di \(A\) con \(B\). \\
\(A\) prende il nome di \emph{categoria sintattica} o \emph{simbolo non terminale}. \\
Proviamo ad esprimere l'espressione \(5 * 3\) utilizzando le produzioni che abbiamo appena definito:
\begin{align*}
  E & = E * E \\
    & = N * E \\
    & = N * N \\
    & = 5 * N \\
    & = 5 * 3
\end{align*}
\subsection*{Le derivazioni}
Una derivazione è il processo passo-passo che identifica le "frasi" legali in una grammatica.\\
Quindi, data la grammatica: \(G = \angled{N,\Sigma,P,S}\) e
\begin{itemize}
  \item \(\delta \in (N \cup \Sigma)^+\), una stringa di caratteri terminali e non terminali \textbf{non vuota}
  \item \(\gamma \in (N \cup \Sigma)^*\), una stringa di caratteri terminali e non terminali
\end{itemize}
definiamo
\begin{itemize}
  \item \textbf{Derivato immediato}: \(\gamma \rightarrow \delta\).\\
        \(\delta\) è derivato immediato di \(\gamma\) se e solo se \(\delta\) si ottiene da \(\gamma\) applicando una sola produzione.
  \item \textbf{Derivato}: \(\gamma \rightarrow^* \delta\).\\
        \(\delta\) è derivato di \(\gamma\) se \(\delta\) si ottiene da \(\gamma\) applicando un numero qualsiasi di produzioni.
\end{itemize}
\subsection*{Derivazioni canoniche}
Esistono due tipi di derivazioni canoniche:
\begin{itemize}
  \item Canonica destra, secondo la quale si espande sempre il simbolo non terminale più a destra
  \item Canonica sinistra, secondo la quale si espande sempre il simbolo non terminale più a sinistra
\end{itemize}
\subsection*{Forma normale di Backus-Naur (BNF)}
Una notazione standard più semplificata. Un esempio di grammatica in BNF
\begin{itemize}
  \item \(E ::= E + E \mid E - E \mid E * E \mid -E \mid (E) \mid V\)
  \item \(V ::= x \mid y \mid z\)
  \item \(I ::= 0 \mid \ldots \mid 9\)
\end{itemize}
\subsection*{Classificazione delle grammatiche}
Le grammatiche possono essere classificate in base alla loro espressività (Gerarchia di Chomsky). I LdP moderni sono grammatiche \emph{libere dal contesto}, ovvero una delle definizioni date da Chomsky, più restrittiva della definizione generale.
\subsubsection*{Grammatiche generali: }
\((\alpha, \beta), \alpha \in (N \cup \Sigma)^+, \beta \in (N \cup \Sigma)^*\)
\subsubsection*{Grammatiche libere dal contesto: }
\((a, \beta), \beta \in (N \cup \Sigma)^*\)
\subsection*{Parse tree (Alberi di derivazione)}
Le derivazioni si possono rappresentare attraverso una struttura matematica ad albero discreta dove:
\begin{itemize}
  \item il nodo radice è etichettato con \(S\)
  \item i nodi sono etichettati con simboli non terminali
  \item le foglie sono i simboli terminali
\end{itemize}
La frontiera di un albero di derivazione, ovvero la sequenza delle sue foglie, rappresenta la sequenza di derivazioni utilizzate.
\subsection*{Alberi di sintassi astratta}
L'analizzatore sintattico (\emph{parser}) utilizza le grammatiche non ambigue per determinare correttamente la precedenza tra operatori. Quindi a partire da una grammatica genera una struttura ad albero che identifica le sequenze di produzioni possibili e definisce una precedenza tra queste ultime. Gli alberi di sintassi astratta sono una semplificazione degli alberi di derivazione che omettono le informazioni ridondanti delle sequenza di simboli non terminali.
\subsection*{Albero di sintassi}
La struttura ad albero viene utilizzata anche per rappresentare in modo formale le stringhe del linguaggio che sono sintatticamente corrette.
\subsection*{Definizione formale di linguaggio utilizzando le grammatiche}
Un linguaggio L generato a partire da una grammatica G è definito come segue.
\begin{dfn}{Linguaggio:}
  L'insieme delle stringhe di caratteri terminali ottenuti applicando un numero qualsiasi di produzioni a partire dal simbolo iniziale.
  \[L(G) = \{ w \mid w \in \Sigma^* \land S \rightarrow^* w\}\]
\end{dfn}
\noindent Due grammatiche sono equivalenti se generano lo stesso linguaggio.
\[L(G) = L(G^\prime)\]
\pagebreak

\subsection{Semantica}
La \emph{semantica} ha il compito di determinare il significato dei programmi sintatticamente corretti, ed è definita come una funzione che associa alle stringhe del linguaggio un significato.
\[f: L \rightarrow \mathbb{N}\]
\begin{itemize}
  \item \(L\) è il dominio sintattico, ovvero l'insieme delle frasi legali del linguaggio.
  \item \(\mathbb{N}\) è il dominio semantico, ovvero l'interpretazione delle stringhe.
\end{itemize}
Le stringhe del dominio sintattico \underline{non hanno significato} prima di applicare la \(f\).

\vspace{.5cm}

\noindent\underline{Come si può determinare questa funzione quindi?}

Se \(L\) fosse un insieme finito, basterebbe una sorta di tabella di corrispondenza. Ma nel nostro caso l'insieme delle stringhe generabili da una grammatica è infinito.

Utilizzeremo quindi il concetto di \emph{composizionalità}, che in parole povere da un significato agli elementi atomici e delle regole per comporre questi elementi, che daremo nel dominio semantico. Assegnando un significato ai simboli del linguaggio e definiamo un modo per comporre i significati si riesce a definire la semantica attraverso elementi finiti, che però sono in grado di interpretare le infinite frasi del linguaggio.
%illustrazione composizionalità
\subsection*{Astrazione}
Alla base dell'analisi semantica troviamo una serie di definizioni che ci premettono di astrarre i concetti di programmazione e di esecuzione del codice.

\begin{dfn}{Identificatori: }
  Sono stringhe che danno un nome significativo al dato che rappresentano.
  Seguono alcune convenzioni e non possono cominciare con un numero.
\end{dfn}

\begin{dfn}{Variabili: }
  Sono identificatori che individuano delle \emph{locazioni di memoria} il cui contenuto può essere variato nel corso del programma.
\end{dfn}
\begin{dfn}{Costanti: }
  Sono identificatori che individuano dei valori che non cambiano per tutto il corso del programma.
\end{dfn}

\begin{dfn}{Assegnamento: }
  Un operatore che serve per scrivere un valore all'interno di un locazione di memoria. Prima valuta la parte destra (\emph{rvalue}) e poi scrive il risultato della valutazione nella parte sinistra (\emph{lvalue}).
\end{dfn}

\begin{dfn}{Macchina astratta: }
  Per analizzare al meglio la \textbf{bontà} di una soluzione definiamo una \emph{macchina astratta}, ovvero l'astra\-zione di un'architettura generica. Il modello a cui faremo riferimento è quello di Von-Neumann che descrive: \emph{memoria}, \emph{controllo di flusso} e \emph{CPU}.
  Definiamo inoltre un meccanismo di esecuzione chiamato ciclo \emph{fetch-execute}.
  %illustr
\end{dfn}

\pagebreak

\subsection{Semantica statica}
La semantica statica si occupa di analizzare i tipi all'interno del programma da eseguire. L'analiz\-zatore semantico riceve in input un albero di sintassi astratta, ovvero un programma corretto per le regole della grammatica utilizzata. \\
In analisi statica ho bisogno quindi di distinguere variabili, costanti e tipi di dato, per farlo utilizzo la funzione \textbf{ambiente statico}
\[\Delta: Id \cup Val \rightarrow T \cup TLoc\]
Nelle espressioni le stringhe di cui tratteremo vengono considerate come una sequenza di \emph{letterali}, ovvero costanti o valori, composti mediante operatori unari o binari. \\
Per arrivare alla conclusione che una stringa del programma è semanticamente ben formata serve un meccanismo finito in grado di considerare le infinite espressioni generabili secondo le regole di sintassi. Sfruttiamo il principio di composizionalità e definiamo la semantica delle produzioni della grammatica che generano le nostre espressioni. In questo modo possiamo comporre la semantica per tutte le possibili espressioni generabili.
\subsection*{Espressioni}
Un'espressione \(E\) è ben formata se partendo dall'ambiente statico \(\Delta\) posso associare ad \(E\) il tipo del valore che rappresenta.
\subsubsection*{Assiomi}
\begin{itemize}
  \item (A1): \(\varnothing \vdash_e i : Int\)
  \item (A2): \(\varnothing \vdash_e d : Double\)
  \item (A3): \(\varnothing \vdash_e b : Bool\)
  \item (A4): \(\varnothing \vdash_e s : String\)
\end{itemize}
\subsubsection*{Regole di inferenza}
\begin{itemize}
  \item (R1)
        \[\frac
          {\Delta(Id) = \tau \lor \Delta(Id) = \tau Loc}
          {\Delta \vdash_e Id:\, \tau}\]

  \item (R2)
        \[\frac
          {\Delta \vdash_e E_1:\, \tau,\ uop: \tau_1 \rightarrow \tau}
          {\Delta \vdash_e uop\,E_1 \,:\, \tau}\]

  \item (R3)
        \[\frac
          {\Delta \vdash_e E_1:\, \tau,\ \Delta \vdash_e E_2:\, \tau,\ bop: \tau_1 \times \tau_2 \rightarrow \tau}
          {\Delta \vdash_e E_1\,bop\,E_2 \,:\, \tau}\]

  \item (R4)
        \[\frac{\Delta \vdash_e E:\, \tau}{\Delta \vdash_e (E):\, \tau}\]
\end{itemize}
\subsection*{Comandi}
Un comando \(C\) è ben formato se in un assegnamento riesco ad associare un tipo all'identificatore e all'espressione sulla destra
\subsubsection*{Assiomi}
\begin{itemize}
  \item (A5): \(\varnothing \vdash_c nil\)
\end{itemize}
\subsubsection*{Regole di inferenza}
\begin{itemize}
  \item (R5)
        \[\frac
          {\Delta \vdash_e E:\, \tau,\ \Delta(Id) = \tau Loc}
          {\Delta \vdash_c Id = E}\]

  \item (R6)
        \[\frac
          {\Delta \vdash_c C_1,\ \Delta \vdash_c C_2}
          {\Delta \vdash_c C;C}\]

  \item (R7)
        \[\frac
          {\Delta \vdash_e E:\, Bool,\ \Delta \vdash_c C}
          {\Delta \vdash_c \text{ if($E$)\{$C$\}}}\]

  \item (R8)
        \[\frac
          {\Delta \vdash_e E:\, Bool,\ \Delta \vdash_c C}
          {\Delta \vdash_c \text{ while($E$)\{$C$\}}}\]

  \item (R9)
        \[\frac
          {\Delta \vdash_d D:\, \Delta',\ \Delta[\Delta'] \vdash_c C}
          {\Delta \vdash_c D;C}\]
\end{itemize}
\subsection*{Dichiarazioni}
Una dichiarazione \(D\) è ben formata se rispetta i tipi e mi permette di produrre un nuovo ambiente statico \(\Delta'\)
\subsubsection*{Assiomi}
\begin{itemize}
  \item (A6) \[\varnothing \vdash_d nil \,:\, \varnothing\]
\end{itemize}
\subsubsection*{Regole di inferenza}
\begin{itemize}
  \item (R10)
        \[\frac
          {\Delta \vdash_e E:\, \tau,\ T == \tau}
          {\Delta \vdash_d \text{const}\ Id:T = E}\]

  \item (R11)
        \[\frac
          {\Delta \vdash_e E:\, \tau,\ T == \tau}
          {\Delta \vdash_d \text{var}\ Id:T = E \,:\, [(Id, \tau Loc)]}\]

  \item (R12)
        \[\frac
          {\Delta \vdash_d D_1:\, \Delta_1,\ \Delta[\Delta_1] \vdash_d D_2:\, \Delta_2}
          {\Delta \vdash_d D_1;D_2:\, \Delta[\Delta_1][\Delta_2]}\]
\end{itemize}
\subsection*{Funzioni}
\begin{itemize}
  \item (FS1)
        \[\ddfrac
          {\Delta \vdash_e E : \tau}
          {\Delta \vdash_c \text{return}\ E;}
        \]
  \item (FS2)
        \[\ddfrac
          {
            form : \Delta_0,\
            \Delta[\Delta_0] \vdash_c C; \text{return}\ E
          }{
            \Delta \vdash_d (\text{function}\ Id(form) \rightarrow \tau \{C;\text{return}\ E\}) \,:\, [(Id, \mathcal{F}(form) \rightarrow \tau)]
          }
        \]
  \item (FS3)
        \[
          nil : \varnothing,\
          \ddfrac{
            form : \Delta_0,\ Id \notin \Delta_0
          }{
            (\text{const}\ Id:\tau, form) \,:\, \Delta_0[(Id, \tau)]
          },\
          \ddfrac{
            form : \Delta_0,\ Id \notin \Delta_0
          }{
            (\text{var}\ Id:\tau, form) \,:\, \Delta_0[(Id, \tau Loc)]
          }
        \]
  \item (FS4)
        \[\ddfrac{
            \Delta \vdash_{ae} ae : aet, \Delta(Id) = aet \rightarrow \tau
          }{
            \Delta \vdash_e Id(ae) : \tau
          }
        \]
\end{itemize}
\pagebreak

\subsection{Semantica dinamica}
La semantica dinamica definisce una sequenza di transizioni tra stati della macchina astratta. La semantica dinamica simula l'esecuzione di un programma ignorando i tipi perché sono già verificati dalla semantica statica, lavorando quindi su valori. \\
In semantica dinamica un programma è considerato come una serie di istruzioni atomiche che modificano gli stati, ovvero una rappresentazione di un istante di tempo durante l'esecuzione. Uno stato è l'insieme delle istruzioni ancora da eseguire associato ad una istanza dell'ambiente e della memoria.
\begin{dfn}{Ambiente: }
  \(\rho: Id \rightarrow Loc \cup Val\)
\end{dfn}
\begin{dfn}{Memoria: }
  \(\sigma: Loc \rightarrow Val\)
\end{dfn}
L'analisi dinamica segue un processo ben preciso che partendo da uno stato iniziale \quoted{calcola} il risultato finale dell'esecuzione di un certo codice. Ogni esecuzione è articolata in 3 fasi:
\begin{itemize}
  \item L'esecuzione comincia dallo \emph{stato iniziale} dove tutte le istruzioni del programma sono ancora da eseguire.
  \item Per passare da uno stato all'altro si applicano le regole di semantica statica definite per \textbf{comandi} (C), \textbf{espressioni} (E) e \textbf{dichiarazioni} (D).
  \item Alla fine della \emph{simulazione} il risultato del programma si troverà nello stato finale.
\end{itemize}
\subsection*{Espressioni}
La semantica dinamica delle espressioni simula la valutazione delle espressioni all'interno del programma.
\[\angled{E,\rho,\sigma} \longrightarrow_e \angled{E',\rho',\sigma'}\]
\[Eval(E,\rho,\sigma) = v \in Val \iff \angled{E,\rho,\sigma} \longrightarrow_e^* v\]
\subsubsection*{Regole di inferenza}
\begin{itemize}
  \item (id1)
        \[\frac
          {\rho(Id) = v \lor (\rho(Id) = L \in Loc \land \sigma(L) = v)}
          {\angled{Id,\rho,\sigma} \longrightarrow_e v}\]

  \item (uop1)
        \[\frac
          {\angled{E,\rho,\sigma} \longrightarrow_e \angled{E',\rho,\sigma}}
          {\angled{uop\,E,\rho,\sigma} \longrightarrow_e \angled{uop\,E',\rho,\sigma}}\]
  \item (uop2)
        \[\angled{E,\rho,\sigma} \longrightarrow_e v' = \text{uop } v\]

  \item (bop1)
        \[\frac
          {\angled{E_1,\rho,\sigma} \longrightarrow_e \angled{E_1',\rho,\sigma}}
          {\angled{E_1 \,bop\, E_2,\rho,\sigma} \longrightarrow_e \angled{E_1'\,bop\,E_2,\rho,\sigma}}\]
  \item (bop2)
        \[\frac
          {\angled{E_2,\rho,\sigma} \longrightarrow_e \angled{E_2',\rho,\sigma}}
          {\angled{v_1\,bop\,E_2,\rho,\sigma} \longrightarrow_e \angled{v_1\,bop\,E_2',\rho,\sigma}}\]
  \item (bop3)
        \[\angled{v_1\,bop\,v_2,\rho,\sigma} \longrightarrow_e v = v_1 \text{ bop } v_2\]
\end{itemize}
\subsection*{Comandi}
\[\angled{C,\rho,\sigma} \longrightarrow_c \angled{C',\rho',\sigma'},\ Exec(C,\rho,\sigma) = \sigma' \iff \angled{C,\rho,\sigma} \longrightarrow_c^* \sigma'\]
\subsubsection*{Regole di inferenza}
\begin{itemize}
  \item (id2)
        \[\frac
          {\angled{E,\rho,\sigma} \longrightarrow_c^* v}
          {\angled{Id = E,\rho,\sigma} \longrightarrow_c \angled{Id = v,\rho,\sigma}}\]
  \item (id3)
        \[\angled{Id = E,\rho,\sigma} \longrightarrow_c \sigma[\rho(Id) = v]\]

  \item (seq1)
        \[\frac
          {\angled{C_1,\rho,\sigma} \longrightarrow_c \angled{C_1',\rho,\sigma}}
          {\angled{C_1;C_2,\rho,\sigma} \longrightarrow_c \angled{C_1';C_2,\rho,\sigma'}}\]
  \item (seq2)
        \[\frac
          {\angled{C_1,\rho,\sigma} \longrightarrow_c \sigma'}
          {\angled{C_1;C_2,\rho,\sigma} \longrightarrow_c \angled{C_2,\rho,\sigma'}}\]

  \item (if1)
        \[\frac
          {\angled{E,\rho,\sigma} \longrightarrow_e^* true}
          {\angled{\text{if($E$)\{$C_1$\} else \{$C_2$\}},\rho,\sigma} \longrightarrow_c \angled{C_1,\rho,\sigma}}\]
  \item (if2)
        \[\frac
          {\angled{E,\rho,\sigma} \longrightarrow_e^* false}
          {\angled{\text{if($E$)\{$C_1$\} else \{$C_2$\}},\rho,\sigma} \longrightarrow_c \angled{C_2,\rho,\sigma}}\]

  \item (rep1)
        \[\frac
          {\angled{E,\rho,\sigma} \longrightarrow_e^* true}
          {\angled{\text{while(E)\{C\}},\rho,\sigma} \longrightarrow_c \angled{C;\text{while($E$)\{$C$\}},\rho,\sigma}}\]
  \item \((rep2)\)
        \[\frac
          {\angled{E,\rho,\sigma} \longrightarrow_e^* false}
          {\angled{\text{while(E)\{C\}},\rho,\sigma} \longrightarrow_c \sigma'}\]
\end{itemize}
\subsection*{Dichiarazioni}
\[\angled{D,\rho,\sigma} \longrightarrow_e \angled{D',\rho',\sigma'},\ Elab(D,\rho,\sigma) = \angled{\rho',\sigma'} \iff \angled{D,\rho,\sigma} \longrightarrow_d^* (\rho',\sigma')\]
\subsubsection*{Regole di inferenza}
\begin{itemize}
  \item (const1)
        \[\frac
          {\angled{E,\rho,\sigma} \longrightarrow^*_e v}
          {\angled{\text{const} Id: T = E,\rho,\sigma} \longrightarrow_d \angled{[(Id,v)],\sigma}}\]

  \item (var1)
        \[\frac
          {\angled{E,\rho,\sigma} \longrightarrow^*_e v}
          {\angled{\text{var} Id: T = E,\rho,\sigma} \longrightarrow_d \angled{[(Id,\text{new} L)],[(L,v)]}}\]

  \item (dd1)
        \[\frac
          {\angled{D_1,\rho,\sigma} \longrightarrow^*_d \angled{D_1',\rho',\sigma'}}
          {\angled{D_1;D_2,\rho,\sigma} \longrightarrow_d \angled{D_1';D_2,\rho',\sigma'}}\]

  \item (dd2)
        \[\frac
          {\angled{D_2,\rho[\rho_1],\sigma} \longrightarrow^*_d \angled{D_2',\rho[\rho_1]',\sigma'}}
          {\angled{\rho_1;D_2,\rho[\rho_1],\sigma} \longrightarrow_d \angled{\rho_1';D_2,\rho[\rho_1]',\sigma'}}\]

  \item (dd3)
        \[\angled{\rho_1,\rho_2,\rho,\sigma} \longrightarrow_d \angled{\rho_1[\rho_2], \sigma}\]
\end{itemize}
\pagebreak

\subsection{Scoping}
Il campo di azione di un identificatore, detto scope, è la porzione di programma in cui l'identificatore può essere utilizzato. \emph{Block-Scope} significa che gli identificatori possono essere utilizzati nel blocco di programma in cui sono definiti, e in eventuali \emph{sotto-blocchi}.
Gli identificatori dichiarati sono \textbf{identificatori legati} e le dichiarazioni sono \textbf{occorrenze di legame}. Esistono due tipi di scoping
\begin{itemize}
  \item Lo \textbf{Scoping statico} si basa sull'albero di sintassi e non richiede l'esecuzione del codice per controllare la sua validità.
  \item Lo \textbf{Scoping dinamico} analizza il codice a tempo di esecuzione stabilendo i legami attraverso l'ambiente.
\end{itemize}
Possiamo quindi partizionare gli identificatori in \textbf{liberi} e \textbf{legati} in base alle regole di scoping.
\pagebreak

\subsection{Funzioni}
Le funzioni ci permettono di applicare il principio di astrazione, ovvero di ridurre la duplicazione di informazione nei programmi, facilitndo la comprensione e menutenzione del codice. Le funzioni consentono
\begin{itemize}
  \item modularità
  \item chiarezza e leggibilità
  \item fattorizzazione del codice
  \item separazione tra utilizzo e implementazione
\end{itemize}
Nell'utilzzo delle funzioni, distinguiamo due fasi diverse
\begin{itemize}
  \item \textbf{definizione di funzione}: definizione del codice che realizza l'operazione astratta
  \item \textbf{chiamata di funzione}: utilizzo della funzione attraverso un espressione
\end{itemize}
\subsection*{Definizione di funzione}
\subsection*{Chiamata di funzione}
La chiamata di funzione è un espressione che viene valutatà così
\begin{enumerate}
  \item viene sospesa l'esecuzione del chiamante e vengono assegnati i parametri attuali a quelli formali
  \item esecuzione di F
  \item istruzione return che termina l'esecuzione di F
  \item il controllo ritorna al chiamante, e la chiamata ad F assume il valore \quoted{calcolato} dal return
\end{enumerate}

\pagebreak

\section{Calcolo della complessità}
\subsection{Limiti asintotici}
Usiamo la notazione asintotica per dare un limite ad una funzione \(f(n)\) a meno di un fattore costante \(c\). In sostanza vogliamo trovare una funzione che sta sempre sopra (limite \emph{superiore}) o sempre sotto (limite \emph{inferiore}) ad \(f(n)\).\\
Nell analizzare un algoritmo utilizzeremo il limite inferiore asintotico per dare un limite di tempo minimo per un qualunque input di dimensione N. Viceversa, utilizzeremo il limite superiore asintotico per dare un limite di tempo massimo per un qualunque input di dimensione N.
\subsubsection*{Limite superiore asintotico}
Se esistono \(c > 0\), \(n_0 \geq 0\) tali che
\[f(n) \leq c \cdot g(n)\,\forall n \geq n_0\]
\(g(n)\) è detto limite superiore asintotico di \(f(n)\) e diciamo che \(f(n)\) è \textbf{O-grande} di \(g(n)\)
\[f(n) = O\left(g(n)\right)\]
Il limite è \textbf{non stretto} se per ogni \(c > 0\) esiste un \(n_0\) tale che
\[f(n) < c \cdot g(n)\,\forall n \geq n_0\]
quindi scriviamo \(f(n) = o\left(g(n)\right)\) che si legge f è \textbf{o-piccolo} di g.
\subsubsection*{Limite inferiore asintotico}
Se esistono \(c > 0\), \(n_0 \geq 0\) tali che
\[f(n) \geq c \cdot g(n)\,\forall n \geq n_0\]
\(g(n)\) è detto limite inferiore asintotico di \(f(n)\) e diciamo che \(f(n)\) è \textbf{Omega-grande} di \(g(n)\)
\[f(n) = \Omega\left(g(n)\right)\]
Il limite è \textbf{non stretto} se per ogni \(c > 0\) esiste un \(n_0\) tale che
\[f(n) > c \cdot g(n)\,\forall n \geq n_0\]
quindi scriviamo \(f(n) = \omega\left(g(n)\right)\) che si legge f è \textbf{omega-piccolo} di g.
\subsubsection*{Limite asintotico stretto}
Se \(f(n) = O\left(g(n)\right)\) e \(f(n) = \Omega\left(g(n)\right)\), \(g(n)\) è detto limite asintotico stretto di \(f(n)\)
\[f(n) = \Theta\left(g(n)\right)\]
Si legge \(f(n)\) è \textbf{Theta} di \(g(n)\).
\begin{oss}
  \(f(n) = O\left(g(n)\right)\) e \(f(n) = \Omega\left(g(n)\right)\) implica che esistono \(c_1, c_2, n_0 > 0\) tali che
  \[c_1 \cdot g(n) \leq f(n) \leq c_2 \cdot g(n) \quad \forall n \geq n_0\]
\end{oss}
\subsubsection*{Teoremi}
\begin{itemize}
  \item \(f(n) = O\left(g(n)\right)\) se e solo se \(g(n) = \Omega\left(f(n)\right)\)
  \item Se \(f_1(n) = O\left(f_2(n)\right)\) e \(f_2(n) = O\left(f_3(n)\right)\) allora \(f_1(n) = O\left(f_3 (n)\right)\)
  \item Se \(f_1(n) = \Omega\left(f_2(n)\right)\) e \(f_2(n) = \Omega\left(f_3(n)\right)\) allora \(f_1(n) = \Omega\left(f_3 (n)\right)\)
  \item Se \(f_1(n) = \Theta\left(f_2(n)\right)\) e \(f_2(n) = \Theta\left(f_3(n)\right)\) allora \(f_1(n) = \Theta\left(f_3 (n)\right)\)
  \item Se \(f_1(n) = O\left(g_1(n)\right)\) e \(f_2(n) = O\left(g_2(n)\right)\) allora \[O\left(f_1(n) + f_2(n)\right) = O\left(max\{g_1(n), g_2(n)\}\right)\]
  \item Se \(f(n)\) è un polinomio di grado \(d\) allora \[f(n) = \Theta\left(n^d\right)\]
\end{itemize}
\pagebreak

\subsection{Caso pessimo, medio e migliore}
Useremo i limiti asintotici per analizzare il caso pessimo e il caso migliore di esecuzione di un algoritmo, individuando quindi un costo minimo e un costo massimo.
\begin{itemize}
  \item Analisi del caso migliore:
  \item Analisi del caso peggiore:
\end{itemize}
L'analisi del caso medio avviene mediante l'utilizzo della probabilità dove ogni caso ha la stessa probabilità di verificarsi.

\subsection{Complessità di funzioni ricorsive}
\subsection*{Relazione di ricorrenza}
Una relazione di ricorrenza è un equazione o disequazione che descrive una funzione in termini di suoi valori per argomenti più piccoli.
\subsubsection*{Ricorrenza divide-et-imperta}
Il costo degli algoritmi di tipo divide-et-impera si può esprimere come la somma di tre costi diversi
\begin{itemize}
  \item \textbf{Divide:} Se l'istanza del problema da risolvere è troppo \quoted{complicata} per essere risolta direttamente, dividila in due o più sottoproblemi.
  \item \textbf{Impera:} Utilizza sempre la tecnica Divide-et-impera per risolvere i sottoproblemi
  \item \textbf{Combine:} Combina le soluzioni dei sottoproblemi per trovare una soluzione al problema originario.
\end{itemize}
L'equazione di ricorrenza diventa quindi
\begin{equation*}
  T(n) =
  \begin{cases}
    c                                   & \text{se } n \leq k \\
    D(n) + \sum_{i = 1}^h T(n_i) + C(n) & \text{se } n > k
  \end{cases}
\end{equation*}
Una equazione di ricorrenza generalizzata, considerando di dividere il problema iniziale in \(a\) sottoproblemi di dimensione \(n/b\) e impiegare un costo \(D(n)\) per dividere e un costo \(C(n)\) per combinare.
\begin{equation*}
  T(n) =
  \begin{cases}
    \Theta(1)             & \text{se } n \leq k \\
    D(n) + aT(n/b) + C(n) & \text{se } n > k
  \end{cases}
\end{equation*}
\subsection*{Metodo della sostituzione}
Il metodo della sostituzione consiste in due step:
\begin{enumerate}
  \item Indovinare la forma della soluzione
  \item Utilizzare l'induzione per trovare le costanti e mostrare che la soluzione funziona
\end{enumerate}
infatti sostituiamo la nostra ipotesi al posto della funzione nella dimostrazione per induzione.
\subsubsection*{Esempio}
\begin{itemize}
  \item Costo base: \(h(1) = 1\)
  \item Costo per un N generico: \(h(N) = 2h(N - 1) + 1,\quad N > 1\)
\end{itemize}
Se sviluppiamo la formula per il costo generico
\begin{align*}
  h(N) & = 2h(N - 1) + 1 = 2(2h(N - 2) + 1) + 1           \\
       & = 4h(N - 2) + 3 = 4(2h(N - 3) + 1) + 3           \\
       & = 8h(N - 3) + 7                                  \\
       & = 2^{N-1}h\left(N - (N - 1)\right) + 2^{N-1} - 1 \\
       & = 2^{N-1}h(1) + 2^{N-1} - 1                      \\
       & = 2^{N} - 1                                      \\
\end{align*}
riusciamo ad esprimere la complessità eliminando la definizione ricorsiva.
\subsection*{Metodo dell'albero}
Possiamo generare un albero ricorsivo che rappresenta l'esecuzione del nostro algoritmo, dove
\begin{itemize}
  \item I nodi rappresentano il costo dei sottoproblemi
  \item Le foglie rappresentano il costo dei sottoproblemi al caso base
\end{itemize}
Per trovare il costo dell'algoritmo sommiamo i costi di ogni livello per torvare il costo del livello stesso, e poi sommaiamo tutti i costi per ottenere il costo totale.
\pagebreak

\subsection{Master theorem}
Utilizziamo il master theorem per risolvere equazioni di ricorrenza della forma:
\[T(n) = aT\left(\frac{n}{b}\right) + f(n)\]
\subsubsection*{Caso 1.}
Se \(f(n)\) cresce polinomialmente più lentamente di \(n^{\log_b a}\)
\[f(n) = O\left(n^{\log_b a - \epsilon}\right)\]
\begin{center}allora la soluzione è\end{center}
\[T(n) = \Theta\left(n^{\log_b a}\right)\]
\subsubsection*{Caso 2.}
Se \(f(n)\) è \(\Theta\)-grande di \(n^{\log_b a}\lg^k n\) per qualche costante \(k \geq 0\)
\[f(n) = \Theta\left(n^{\log_b a}\lg^k n\right)\]
\begin{center}allora la soluzione è\end{center}
\[T(n) = \Theta\left(n^{\log_b a} \lg^{k+1} n\right)\]
\subsubsection*{Caso 3.}
Se \(f(n)\) cresce polinomialmente più velocemente di \(n^{\log_b a}\)
\[f(n) = \Omega\left(n^{\log_b a + \epsilon}\right)\]
\begin{center}e\end{center}
\[a f(\frac{n}{b}) < c f(n)\]
\begin{center}
  per qualche costante \(c < 1\) e per tutti gli \(n > n_0\) \\
  allora la soluzione è
\end{center}
\[T(n) = \Theta\left(f(n)\right)\]
\subsubsection*{Applicazione master theorem}
\underline{\textbf{ogni risposta va giustificata}}
\begin{enumerate}
  \item Confronto \quoted{a spanne} \(f(n)\) vs \(n^{log_b a}\). Quale cresce più velocemente?
  \item Dimostrazione con definizione di O-grande / \(\Omega\)-grande / \(\Theta\), quindi trovare \(\epsilon\) e \(c\).
  \item Nel terzo caso verificare la condizione di \emph{regolarità}
\end{enumerate}
\pagebreak

\subsection{Note sulla complessità}
\subsection*{Tempo di esecuzione}
Avendo tre algoritmi, uno lineare, uno quadratico e uno cubico, aumentando di k l'efficienza del calcolatore otteniamo un vantaggio pari all'efficienza dell'algoritmo stesso.
\begin{align*} T(n) & = n &  &  & N_l & = kT \\ T(n) & = n^2 & & & N_q & =
               \sqrt[2]{kT}                  \\ T(n) & = n^3 & & & N_c & = \sqrt[3]{kT}\end{align*}
Per questo motivo conviene concentrarsi fin da subito sulla progetta di algoritmi più efficienti piuttosto che attendere calcolatori più potenti in grado di calcolare soluzioni più velocemente.
\subsection*{Limite Inferiore o Lower Bound}
Il \emph{lower bound} di un problema rappresenta il numero minimo di azioni necessarie per risolvere il problema.
E.g.: Per sommare \(n\) numeri ho bisogno di \(n-1\) somme, perché facendo \(n-2\) o meno somme mi \emph{perdo} uno o più numeri e la soluzione è certamente sbagliata.
\pagebreak

\section{Strutture dati}
\subsection{Pile}
Uno stack (o pila) è un insieme dinamico dove l'elemento ad essere rimosso ad una cancellazione è quello inserito per ultimo, secondo lo schema \textbf{LIFO} (\emph{Last-In, First-Out}). \\
Negli stack l'operazione di inserimento è detta \emph{PUSH} e quella di cancellazione è detta \emph{POP}.
\subsection{Code}
Una queue (o coda) è un insieme dinamico dove l'elemento ad essere eliminato ad una cancellazione è quello inserito per primo, secondo lo schema \textbf{FIFO} (\emph{First-In, First-Out}). \\
Nelle code l'operazione di inserimento è detta \emph{ENQUEUE} e quella di cancellazione è detta \emph{DEQUEUE}.
\subsection{Alberi}
Un albero è un grafo connesso aciclico. I nodi di grado 1 sono detti foglie, i nodi di grado \(> 1\) sono detti nodi interni.
\begin{itemize}
  \item Un albero si dice \emph{radicato} se ha un nodo \textbf{radice}.
  \item I nodi interni hanno uno o più filgi, che consideriamo ordinati e numerati da \(0\) a \(k\).
\end{itemize}
\subsection*{Alberi binari}
Un albero \textbf{binario} è un'albero k-ario con \(k = 2\). In questo caso i filgi si chiamano sinistro (sx) e destro (dx).
\subsubsection*{Proprietà degli alberi binari}
\begin{itemize}
  \item Ci sono al più \(2^t\) nodi al livello \(l\) di un albero binario
  \item Un albero binario di profondità \(d\) ha al più \(2^{d+1}-1\) nodi
  \item Un albero binario con \(n\) nodi ha profondità \(\log_2(n+1)-1\) (la base del \emph{log} corrisponde alla arietà dell'albero)
\end{itemize}
\pagebreak

\section{Algoritmi di ordinamento 1}
L'ordinamento è una procedura tale che data una sequenza di numeri restituisce la sua permutazione ordinata.
\begin{itemize}
  \item Input: \(\angled{a_1, a_2, \ldots, a_n}\)
  \item Output: \(\angled{a_1', a_2', \ldots, a_n'}\) tale che \(a_1' \leq a_2' \leq \ldots \leq a_n'\)
\end{itemize}
\subsection{Insertion-sort}
L'insertion sort è un algoritmo efficiente per l'ordinamento di poche chiavi. Il concetto fondamentale è simile a come riordineremmo una mano di carte. Si comincia con tutte le carte a faccia in giù. Poi ogni carta pescata viene posizionata corretta nella mano. Se non abbiamo carte la prendiamo in mano e basta, se in mano abbiamo già delle carte, troviamo la posizione corretta partendo da destra e scorrendo le carte e comparando il loro valore con quello della carta pescata. \\
L'insertion-sort è un algoritmo di sorting \emph{in-place}.
\begin{lstlisting}
INSERTION-SORT(A)
1. for j = 2 to A.length
2.   key = A[j]
3.   // Insert A[j] into the sorted sequence A[1..j - 1].
4.   i = j - 1
5.   while i > 0 and A[i] > key
6.     A[i + 1] = A[i]
7.     i = i - 1
8.   A[i + 1] = key
\end{lstlisting}
\subsection{Selection-sort}
Il selection sort consiste nello spostare in cima l'elemento minore della parte restante dell'array, ovvero la parte che ancora non è stata toccata, fino ad ottenere un array ordinato.
\subsection{Merge-sort}
L'argorimo \emph{merge-sort} è di tipo \textbf{divide-et-impera}, che quindi si basa sull'idea di dividere a metà l'array da ordinare. \\
Quindi lo step \emph{divide} seziona l'array di partenza in 2 parti uguali, poi lo step \emph{impera} ordina entrambe le parti, seguendo sempre l'argorimo \emph{merge-sort}, e infine i due array ordianti vengono \emph{fusi} insieme.
\begin{center}
  \includegraphics[width=0.8\textwidth]{merge_sort.png}
\end{center}
\begin{lstlisting}[mathescape, numbers=left]
MERGE(A, p, q, r)
  n1 = q - p + 1
  n2 = r - q
  let L[1..n1 + 1] and R[1..n2 + 1] be new arrays
  for i = 1 to n1
    L[i] = A[p + i - 1]
  for i = 1 to n2
    R[i] = A[q + i]
  L[n1 + 1] = $\infty$
  R[n2 + 1] = $\infty$
  i = 1
  j = 1
  for k = p to r
    if L[i] $\leq$ R[j]
      A[k] = L[i]
      i = i + 1
    else
      A[k] = R[j]
      j = j + 1

MERGE-SORT(A,p,r)
if p < r
  q = $\lfloor$(p + r) / 2$\rfloor$
  MERGE-SORT(A, p, q)
  MERGE-SORT(A, q + 1, r)
  MERGE(A, p, q, r)
\end{lstlisting}
\subsection{Quick-sort}
\begin{lstlisting}[numbers=left]

\end{lstlisting}
Quanto costa il QuickSort?
\begin{itemize}
  \item Caso migliore: il partizione mi costa \(\Theta(n)\), e il pivot mi divide esattamente a metà l'array in due parti \(2T(n/2)\). Con il master theorem (secondo caso) ottengo un costo di \(\Theta(n \log n)\).
  \item Caso peggiore (array ordinato): il partizione mi costa sempre \(\Theta(n)\), ma il pivot mi divide l'array in una parte che composta da un elemento ed un'altra composta da \(n-1\) elementi. Quindi la mia chiamata ricorsiva sarà \(
        T(n-1) + T(0)\), dove ho \(T(0)\) perché il pivot è l'ultimo elemento e quindi la seconda parte è vuota. Se svolgo l'equazione di ricorrenza ottengo \(\Theta(n^2)\).
\end{itemize}

\subsection{Sommario delle complessità}
\begin{center}
  \includegraphics[width=0.8\textwidth]{costo_algo}
\end{center}
\pagebreak

\section{Strutture dati 2}
\subsection{Albero binario di ricerca}
Un albero binario di ricerca è un albero che soddisfa le proprietà
\begin{itemize}
  \item tutti i nodi del sottoalbero sinistro di ogni nodo x hanno la chiave minore o uguale di quella di x
  \item tutti i nodi del sottoalbero destro di un ogni nodo x hanno la chiave maggiore o uguale di quella di x
\end{itemize}
Una visita simmetrica su un ABR ci permette di avere un array di valori tutti ordinati.
\subsection*{Algoritmi su ABR}
\begin{lstlisting}
RICERCA_ABR(T, k)
  if (T == NULL || T.key == k) return T
  if (k < T.key) return Ricerca_ABR(T.left, k)
  else return Ricerca_ABR(T.right, k)
\end{lstlisting}
\begin{lstlisting}
RICERCA_ABR_iter(T, k)
  while T != NULL && T.key != k
    if (k < T.key) T = T.left
    else T = T.right
  return T
\end{lstlisting}
\begin{lstlisting}
TROVA_MIN_ABR_iter(T)
  while T.left != NULL
    T = T.left
  return T
\end{lstlisting}
\begin{lstlisting}
  TROVA_MAX_ABR_iter(T)
    while T.right != NULL
      T = T.right
    return T
\end{lstlisting}
\subsection*{Successore e predecessore}
Dato un nodo x in un ABR possiamo trovare un successore e un predecessore.\\
Il \textbf{successore} si trova prendendo il minimo del sottoalbero destro, se tale sottoalbero esiste, altrimenti prendendo l'antenato più basso il cui figlio sinistro è antenato di x.
Il \textbf{predecessore} si trova prendendo il massimo del sottoalbero sinistro, se tale sottoalbero esiste, altrimenti prendendo l'antenato più basso il cui figlio destro è antenato di x.
\begin{lstlisting}
SUCCESSORE(T)
  if T.right != NULL return MIN(T.right)
  a = T.parent
  while a != NULL && T == a.right
    T = a
    a = a.parent
  return a

PREDECESSORE(T)
  if T.left != NULL return MAX(T.left)
  a = T.parent
  while a != NULL && T == a.left
    T = a
    a = a.parent
  return a
\end{lstlisting}

\begin{lstlisting}
INSERT(T, n)
  y = NULL; x = T
  while x != NULL
    y = x
    if (n.key < x.key) x = x.left
    else x = x.right
    
\end{lstlisting}
\pagebreak

\subsection{Heap}
Un \textbf{Heap} è un albero binario quasi completo i cui nodi vengono piazzati per livelli da sinistra verso destra.
\subsection*{Proprietà}
\begin{itemize}
  \item Albero binario quasi completo con foglie ammassate a sinistra
  \item \textbf{max-heap}: La chiave di ogni nodo \(\geq\) chiavi dei figli.
  \item \textbf{min-heap}: La chiave di ogni nodo \(\leq\) chiavi dei figli.
\end{itemize}
\subsection*{Implementazione}
Un \textbf{Heap} può essere implementato con un array, evitando puntatori. Conoscendo l'indice di un nodo:
\begin{itemize}
  \item Indice del padre: \(\lfloor i / 2 \rfloor\)
  \item Indice del filgio sx: \(i \cdot 2\)
  \item Indice del filgio dx: \(i \cdot 2 + 1\)
\end{itemize}
\pagebreak
\subsection*{Algoritmi su heap}
\subsubsection*{Inserimento}
\begin{enumerate}
  \item Inserire il nodo come foglia ammassando a sinistra
  \item Far risalire il nodo facendo \textbf{swap} col padre finché la chiave del nodo inserito \(\geq\) la chiave del suo padre attuale
\end{enumerate}
\subsubsection*{Rimuovere la radice}
\begin{enumerate}
  \item elimino la radice e sposto l'ultimo nodo al suo posto (ultima riga più a destra)
  \item Scambio il nodo che ha preso il posto di radice con il suo figlio più grande, finche non ho che entrambi i figli sono più piccoli
\end{enumerate}
\subsubsection*{MAX-HEAPIFY}
Mantiene la proprietà di \emph{max-heap} in tempo \(O(n \log n)\).
\begin{verbatim}
MAX-HEAPIFY(A, i)
 1. l = LEFT(i)
 2. r = RIGHT(i)
 3. if l <= A.heap_size && A[l] >= A[i]
 4.   largest = l
 5. else
 6.   largest = i
 7. if r <= A.heap_size && A[r] >= A[largest]
 8.   largest = r
 9. if largest != i
10.   swap(A[i], A[largest])
11.   MAX-HEAPIFY(A, largest)
\end{verbatim}
\subsubsection*{BUILD-MAX-HEAP}
Costruisce un heap in tempo \(O(n)\).
\begin{verbatim}
BUILD-MAX-HEAP(A)
1. n = A.length
2. for(i = n/2, i >= 1, i--)
3.   MAX-HEAPIFY(A, i, n)
\end{verbatim}
\subsubsection*{HEAPSORT}
Ordina un array in tempo \(O(n \log n)\)
\begin{verbatim}
HEAPSORT(A)
1. BUILD-MAX-HEAP(A)
2. while A.heap_size > 1
3.   swap(A[1], A[A.heap_size])
4.   A.heap_size--
5.   MAX-HEAPIFY(A,1)
\end{verbatim}
\pagebreak

\subsection{Priority Queue}
\subsection*{Proprietà}
\begin{itemize}
  \item Ogni elemento è associato ad ua valore (priorità)
  \item La chiave con la più alta (o bassa) proprotà è estratta prima
\end{itemize}
\subsection*{Operazioni}
\begin{itemize}
  \item \textbf{INSERT(S,x)} inserisce x in S
  \item \textbf{EXTRACT-MAX(S)} rimuove e restituisce l'elemento con chiave più grande
  \item \textbf{MAXIMUM(S)} restituisce l'elemento con chiave massima
  \item \textbf{INCREASE-KEY(S,x,k)} imposta a k la chiave dell'elemento x
\end{itemize}
\pagebreak

\subsection{Hash table}
\subsection*{Rehashing}
Se la tabella si riempe troppo conviene aumentare la dimensione della tabella stessa e fare un \emph{rehash} dei valori per mantenere un tempo di accesso costante
\pagebreak

\subsection{2-3 Alberi}
I 2-3 alberi sono alberi binari o ternari completi.
\subsection*{Struttura}
\begin{itemize}
  \item Ogni nodo ha almeno 2 figli e al massimo 3.
  \item Tutte le foglie sono allo stesso livello, quindi si tratta di un'albero completamente bilanciato.
  \item L'altezza di un 2-3 albero è compresa tra \(log_2(n)\) e \(log_3(n)\) dove \(n\) indica il numero di nodi.
\end{itemize}
Formalmente possiamo esprimere la struttura dei 2-3 alberi con il seguente \emph{Lemma}
\begin{enumerate}
  \item \(\displaystyle 2^{h+1} - 1 \leq b \leq \frac{3^{h+1} - 1}{2}\)
  \item \(\displaystyle 2^h \leq f \leq 3^h\)
\end{enumerate}
Dove
\begin{enumerate}
  \item \(n\) indica il numero di nodi
  \item \(f\) indica il numero di foglie
  \item \(h\) indica l'altezza dell'albero
\end{enumerate}
\begin{center}
  \textbf{|dimostrazione per induzione|}
\end{center}
Possiamo quindi affermare che l'altezza di un 2-3 albero è \(\Theta(\log n)\)
\begin{itemize}
  \item I dati sono memorizzati tutti nelle foglie in ordine \textbf{crescente} da sinistra a destra.
  \item I nodi interni non contengono chiavi, bensì memorizzano informazioni supplementari per operazioni di visita e ricerca.
        \begin{itemize}
          \item \(S[v]\) contiene la chiave massima nel sottoalbero del primo figlio di \(v\)
          \item se \(v\) ha 3 figli, \(M[v]\) contiene la chiave massima nel sottoalbero del secondo figlio di \(v\)
        \end{itemize}
\end{itemize}
\begin{center}
  \includegraphics[width=0.8\textwidth]{2-3_alberi.png}
\end{center}
\begin{lstlisting}
  Search(T, x)
\end{lstlisting}
\subsection*{Inserimento}
L'inserimento e eseguito in due fasi, prima localizzo la posizione dove inserire un nuovo nodo \(e\) indetificando un certo \(v\) al penultimo livello, future genitore di \(e\). \\
Ora devo inserire, posso incontrare due casi
\begin{enumerate}
  \item \(v\) ha solo due figli, facile aggiungo e basta.
  \item \(v\) ha già tre filgi, quindi eseguo uno \textbf{split}
\end{enumerate}
\textbf{Split}
\begin{enumerate}
  \item seleziono le due folgie minime che assegnero ad un nuovo \(w\)
  \item le due foglie restanti (massime) verranno assegnate a \(v\)
  \item aggiorno gli \textbf{instradatori} di \(v\) e \(w\)
  \item il nuovo \(w\) diventa fratello di \(v\)
\end{enumerate}
A questo punto se il genitore di \(v\) ha quattro figlio, ripeto la procedura ricorsivamente, fino ad arrivare alla radice.
\subsection*{Cancellazione}
\begin{enumerate}
  \item Localizzo la posizione di \(e\) da eliminare, identificando \(v\) genitore di \(e\)
  \item Questa volta ho tre casi
        \begin{enumerate}
          \item Se genitore di \(v\) ha tre figli, facile basta cancellare la foglia
          \item Se genitore di \(v\) ha due filgi, esegueo l'operazione di \textbf{fusione}
        \end{enumerate}
\end{enumerate}
\textbf{Fusione}
\begin{enumerate}
  \item Identifico un \(w\) fratello di \(v\) (che per costruzione avrò \emph{sicuramente})
  \item Se \(w\) ha due filgi, il fratello di \(e\) diventa figlio di \(w\).
  \item Se \(w\) ha tre figli, ???
\end{enumerate}
\pagebreak

\section{Algoritmi di ordinamento 2}
\subsection*{Ordinamento stabile}
Un algoritmo di ordinamento è \textbf{stabile} se preserva l'ordine iniziale tra due elementi con la stessa chiave.

\subsection{Ordinamento senza confronti}
A volte possiamo sfruttare alcune proprietà note a priori degli oggetti da ordinare per effettuare ordinamenti in tempo lineare.
\pagebreak

\section{Programmazione dinamica}
La programmazione dinamica è una strategia di risoluzione che si propone come alternativa più efficiente rispetto alla strategia divide-et-impera nell'ambito di alcuni problemi di natura ricorsiva. \\
La programmazione dinamica propone una soluzione iterativa che
\begin{itemize}
  \item utilizza meno spazio perché non si è costretti a creare record di attivazione per ogni chiamata
  \item conclude in tempi più brevi perché usa un approccio \textbf{bottom-up}, ovvero trova prima la soluzione dei casi base e risale l'albero dei sottoproblemi salvando i risultati calcolati in una tabella di appoggio detta di \emph{programmazione dinamica}.
\end{itemize}
Ricordiamoci sempre che la soluzione iterativa non è sempre la soluzione più efficiente, in alcuni casi conviene comunque la soluzione ricorsiva.
\subsection*{Quando conviene la programmazione dinamica?}
Quando incotriamo problemi di natura ricorsiva che però contengono ripetizioni di chiamate, ovvero partizionati ma con un \textbf{overlap} dei sottoproblemi, conviene utilizzare la programmazione dinamica.

\subsection{Problema 1: Taglio della corda}
Devo vendere una corda a pezzi. Il prezzo di ogni pezzo dipende dalla sua lunghezza (una relazione che potrebbe non essere lineare). Una volta stabilita una tabella dei prezzi mi chiedo quali lunghezze è più conveniente vendere per \textbf{massimizzare} il guadagno.
\begin{itemize}
  \item Corda di lunghezza \(n\)
  \item Tabeella dei prezzi \(p_i\) con \(i = 1,2,\ldots,n\)
\end{itemize}
Creo un array \(r_i\) che contiene il massimo guadagno per una corda lunga \(i\). L'elemento \(r_n\) contiene il massimo guadagno per l'intera corda, ovvero la soluzione del problema.  \\
La complessità della soluzione iterativa è \(O(n^2)\)

\subsection{Problema 2: Longest common subsequence}
Immaginiamo di avere una traccia dei comandi eseguiti su un certo sistema. Una sotto-sequenza di comandi indetifica un intrusione.
\begin{itemize}
  \item Rappresentiamo i comandi come etichette
  \item Ad esempio la sequenza di intrusione potrebbe essere: \(\angled{A,D,C,A,A,B}\)
\end{itemize}
Formalmente \(S\) è una sotto-sequenza di \(A\) se esistono le posizioni
\begin{center}
  \(0 \leq i_0 < i_1 < \ldots < i_k \leq n - 1\) tali che \\
  \(A[i_j] = S[i_j]\) per \(i = 0, 1, \ldots, k\)
\end{center}
\(S\) è sotto-sequenza \textbf{comune} ad \(A\) e \(B\) se è sotto-sequenza sia di \(A\) che di \(B\)

\subsection*{LCS(A,B) ricorsivamente}
Una chiamata di \(LCS(A,B)\) può essere scomposta nella chiamata ricorsiva \(LCS(A',B')\) dove
\begin{itemize}
  \item \(A' + x = A\) con \(x \in A \setminus A'\)
  \item \(B' + y = B\) con \(y \in B \setminus B'\)
\end{itemize}
Per ricostruire la soluzione al passo ricorsivo, chiamo \(l\) la massima lunghezza della sotto-sequenza comune tra \(A'\) e \(B'\), poi considero le possiblità
\begin{itemize}
  \item Se \(x = y\) allora \(LCS(A,B) = l + 1\)
  \item Se invece \(x \neq y\) ho due scenari
        \begin{itemize}
          \item S1: \(LCS(A, B') = l_1\)
          \item S2: \(LCS(A', B) = l_2\)
          \item Il nuovo valore \(LCS(A,B) = max\{l_1, l_2\}\)
        \end{itemize}
\end{itemize}
Riorganizzando ho che \(LCS(i,j)\) è uguale a
\begin{equation*}
  LCS(i,j) =
  \begin{cases}
    0                                   & \text{se } i = 0 \lor j = 0                \\
    LCS(i - 1, j - 1)                   & \text{se } i,j > 0 \text{ e } x_i = y_j    \\
    max\{LCS(i, j - 1), LCS(i - 1, j)\} & \text{se } i,j > 0 \text{ e } x_i \neq y_j
  \end{cases}
\end{equation*}

\begin{lstlisting}
LCS(A,B):
  for (i = 0; i <= A.length; i++)
    L[i][0] = 0
  for (j = 1; j <= B.length; j++)
    L[0][j] = 0

  for (i = 1; i <= A.length; i++)
    for (j = 1; j <= B.length; j++)
      if (A[i - 1] = B[j - i])
        L[i][j] = L[i - 1][j - 1] + 1
      else if (L[i][j - 1] > L[i - 1][j])
          L[i][j] = L[i][j - 1] + 1
      else 
          L[i][j] = L[i - 1][j] + 1
\end{lstlisting}

\pagebreak

\section*{Prova itinere 03-06}
Esercitazione Mercoledì 03-01.
\begin{itemize}
  \item Semantica statica / dinamica
  \item Heap
  \item Hash
  \item Programmazione dinamica
        \begin{itemize}
          \item Taglio della corda
          \item Longest common subsequence (LCS)
        \end{itemize}
\end{itemize}

\end{document}