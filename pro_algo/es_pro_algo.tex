\documentclass{article}
\usepackage[utf8]{inputenc}
\usepackage[italian]{babel}
\usepackage[margin=1.25in]{geometry}

\usepackage{amsfonts, amssymb}
\usepackage{amsmath, amsthm}
\usepackage[thicklines]{cancel}

\usepackage{graphicx}
\usepackage{hyperref}

\hypersetup{
colorlinks=true,
linkcolor=black,
filecolor=magenta,
urlcolor=blue,
}

\newtheorem*{thm}{Teorema}

\theoremstyle{definition}
\newtheorem*{dfn}{Definizione}
\newtheorem*{oss}{Osservazione}
\newtheorem*{dms}{Dimostrazione}
\newtheorem*{eg}{Esempio}
\newtheorem*{nb}{N.B}

\newcommand{\quoted}[1]{``#1''}
\newcommand{\angled}[1]{\langle #1 \rangle}

\begin{document}
\section{Esercizio 1}
\large\textbf{Input}
\begin{itemize}
    \item Numero intero \(N \geq 1\)
\end{itemize}

\noindent\large\textbf{Output}
\begin{itemize}
    \item Numero di coppie \((i,j)\) tali che \(i\) e \(j\) compresi tra 1 e \(N\)
\end{itemize}

\noindent Codice di base
\begin{center}
    \begin{verbatim}
function Count_1(N: Int)
    sum = 0
    for (i = 1; i <= N; i = i + 1)
        for (j = i; j <= N; j = j + 1)
            sum = sum + 1
    return sum
\end{verbatim}
\end{center}
\textbf{Tempo di esecuzione: }
\[2 + 3N + 1 + \sum_{i = 1}^N \left[3\left(N + 1 - i\right)\right] + 2 \sum_{i = 1}^N \left(N + 1 - i\right)\]

\vspace{1.5cm}

\noindent Posso fare di meglio.
\begin{center}
    \begin{verbatim}
function Count_1(N: Int)
    sum = 0
    for (i = 1; i <= N; i = i + 1)
        sum = sum + (N + 1 - i)
    return sum
\end{verbatim}
\end{center}
\textbf{Tempo di esecuzione: } \(7N + 3\)

\vspace{1.5cm}

\noindent Posso fare (ancora) meglio.
\[\sum_{i=1}^N (N + 1 - i) = \sum_{i = 1}^N i = \frac{N(N + 1)}{2}\]
Quindi posso scrivere:
\begin{center}
    \begin{verbatim}
function Count_1(N: Int)
    return N * (N + 1) / 2
\end{verbatim}
\end{center}
\textbf{Tempo di esecuzione: } \(5\)

\pagebreak

\section{Esercizio 2}
\begin{verbatim}
    var A: [Int] = un array contenente interi;
    var res: Bool = false;
    const n: Int = A.length;
    for(var k: Int = 0; k < sqrt(n); k++)
        for (var h: Int = 0; h < k; h++)
            for (var t: Int = 0; t < h; t++)
                res = res || ((A[t] % 2 == 0) && (A[t+1] % 2 == 1))
\end{verbatim}

\textbf{Dichiarazioni} \\
\textbf{Analisi statica}
\[\Delta = \angled{(A, [Int]Loc), (res, BoolLoc), (n, Int)}\]

\textbf{Analisi dinamica}

Stato 0
\[D_1;D_2;D_3;C\]
\[\rho = \varnothing\]
\[\sigma = \varnothing\]

Stato 1
\[D_2;D_3;C\]
\[\rho = \angled{(A, [Loc])}\]
\[\sigma = \angled{([Loc], [a_1,a_2,\ldots])}\]

Stato 2
\[D_3;C\]
\[\rho = \angled{(A, [Loc]), (res, L_0)}\]
\[\sigma = \angled{([Loc], [a_1,a_2,\ldots]), (L_0, false)}\]


\textbf{Complessità}

Il for più esterno viene eseguito \(sqrt{n}\) volte

Il for intermedio è nell'ordine di \(O(n)\) perchè tramite la formula di Gauss:
\[O\left(\sqrt{n}^2\right) = \frac{\sqrt{n}(\sqrt{n} + 1)}{2}\]

Il for più interno è dell'ordine \(O(n\sqrt{n})\)
\[h = \frac{k\left(k + 1\right)}{2}\]
\[k = \frac{\sqrt{n}\left(\sqrt{n} + 1\right)}{2}\]

\begin{align*}
    s & = \frac{h(h + 1)}{2}                                                         \\
    s & = \frac{\frac{k\left(k + 1\right)}{2}(\frac{k\left(k + 1\right)}{2} + 1)}{2} \\
    s & = \frac{
        \frac{
            \frac{\sqrt{n}\left(\sqrt{n} + 1\right)}{2}
            \left(
            \frac{\sqrt{n}\left(\sqrt{n} + 1\right)}{2} + 1
            \right)
        }
        {2}
        \left(
        \frac{
            \frac{\sqrt{n}\left(\sqrt{n} + 1\right)}{2}
            \left(
            \frac{\sqrt{n}\left(\sqrt{n} + 1\right)}{2} + 1
            \right)
        }{2} + 1
        \right)
    }
    {2}
\end{align*}

\section{Esercizio 3}
Verificare che il programma sia corretto dal punto di vista di semantica statica.

\begin{verbatim}
var xG:Double = 3.5;
if (xG < 3.5) {
    const xL:Double = 13.2;
    var sommaIf:Double = xG + xL;
}
else {
    var sommaElse:Double = xG + xL;
}
sommaIf = sommaIf + 1;
sommaElse = sommaElse + 1;
\end{verbatim}

\begin{enumerate}
    \item \(D_1 = \)\quad var xG:Double = 3.5;
    \item \(D_2 = \)\quad const xL:Double = 13.2;
    \item \(D_3 = \)\quad var sommaIf:Double = xG + xL;
    \item \(D_4 = \)\quad var sommaElse:Double = xG + xL;
    \item \(C_1 = \)\quad 
    \begin{verbatim}
const xL:Double = 13.2
var sommaIf:Double = xG + xL;
    \end{verbatim}
    \item \(C_2 = \)\quad var sommaElse:Double = xG + xL;
    \item \(C_3 = \)\quad
\begin{verbatim}
if (xG < 3.5) {
    const xL:Double = 13.2;
    var sommaIf:Double = xG + xL;
}
else {
    var sommaElse:Double = xG + xL;
}
\end{verbatim}
    \item \(C_4 = \)\quad sommaIf = sommaIf + 1;
    \item \(C_5 = \)\quad sommaElse = sommaElse + 1;
    \item \(C_6 = \)\quad \(C_3;C_4;C_5\)
    \item \(C_7 = \)\quad \(C_4;C_5\)
\end{enumerate}

\[
    \frac{
        \frac{\varnothing \vdash_e 3.5:Double}{\varnothing \vdash_d D_1 : \Delta_1},
        \quad
        \frac
        {
            \Delta_1 \vdash_c C_3,
            \quad
            \frac
            {\Delta_1 \vdash_c C_4, \Delta_1 \vdash_c C_5}
            {\Delta_1 \vdash_c C_7}
        }{\Delta_1 \vdash_c C_6}
    }{\varnothing \vdash_c P}
\]

\end{document}