\documentclass{article}
\usepackage[utf8]{inputenc}
\usepackage[italian]{babel}
\usepackage[margin=1.25in]{geometry}

\usepackage{amsfonts, amssymb}
\usepackage{amsmath, amsthm}
\usepackage[thicklines]{cancel}

\usepackage{graphicx}
\usepackage{hyperref}

\hypersetup{
colorlinks=true,
linkcolor=black,
filecolor=magenta,
urlcolor=blue,
}

\newtheorem*{thm}{Teorema}

\theoremstyle{definition}
\newtheorem*{dfn}{Definizione}
\newtheorem*{oss}{Osservazione}
\newtheorem*{dms}{Dimostrazione}
\newtheorem*{eg}{Esempio}
\newtheorem*{nb}{N.B}

\newcommand{\quoted}[1]{``#1''}
\newcommand{\angled}[1]{\langle #1 \rangle}

\title{Esercitazione 3}
\author{Tiziano Marzocchella [655205]}
\date{A.A. 2022/2023}

\begin{document}
\maketitle

\noindent Esercitazione svolta insieme a Luca Cerretini, Francesco Faenza e Emanuele Chilin

\section{Esercizio 1}
Possiamo osservare che per essere palindorme, le targhe della forma \textbf{XXNNNXX} devono avere le sequenza \(X_1 X_2 N_1\) 
ripetuta simmetricamente alla fine, formando quindi una targa del tipo \(X_1 X_2 N_1 N_2 N_1 X_2 X_1\).
Quindi per contare le targhe palindorme basta trovare il numero di stringhe della forma \(X_1 X_2 N_1 N_2\), che sono esattamente
\[D'_{10,2} \cdot D'_{22,2} = 10^2 \cdot 22^2 = 48400\]

\section{Esercizio 2}
Per contare le possibili configurazioni, mi immagino di disporre le 4 mani di 5 carte ciascuna in sequenza.
Posso quindi trovare le disposizioni di 52 carte del mazzo in classe 20 per trovare le possibili configurazioni di carte che formano le 4 mani di gioco. Manca però da considerare che cambiando l'ordine delle carte nella stessa mano, la configurazione di gioco va considerata uguale, pertanto divido le disposizioni di 52 in classe 20 per le permutazioni di 5 carte per mano elevate alla 4, considerando così le 4 mani di gioco.
\[N_{configurazioni} = \frac{D(52,20)}{P(5)^4}\]

\section{Esercizio 3}
\begin{enumerate}
    \item Volendo trovare i possibili modi in cui 6 persone possono disporsi su 6 posti, mi basta trovare \(P(6) = 6!\) le permutazioni di 6. Volendo considerare tutte le configurazioni cicliche come equivalenti, si può osservare che per ogni permutazione di 6 elementi, posso far scrorrere la configurazione verso destra (o sinistra) fino a sei volte prima di tornare alla configurazione di partenza, in questo modo
    \begin{align*}
        123456 \\
        234561 \\
        345612 \\
        456123 \\
        561234 \\
        612345 \\
        123456
    \end{align*}
    Come si può vedere, per ogni permutazione dei 6 posti, ce ne sono altre 5 equivalenti, ottenute facendo scrorrere la sequenza. La formula finale è quindi
    \[\frac{P(6)}{6} = 6! = 720\]
    \item Per trovare in quanti modo si possono disporre i 6 amici, per prima cosa dobbiamo individuare in quanti modi diversi si possono selezionare i 3 amici che si siederanno al primo dei due tavoli da 3 posti, e i restanti 3 che si siederanno all'altro tavolo vengono individuati di conseguenza.
    \[\binom{6}{3} = \frac{6!}{3!(6 - 3)!}\]
    Poi individuiamo in quanti modi diversi i 6 amici, divisi su due tavoli, si possono disporre
    \[\left(\frac{P(3)}{3}\right)^2\]
    infine bisogna dividere il risulato per due, perchè scambiando i due tavoli dobbiamo considerare la configurazione come equivalente.
    \[\frac{\binom{6}{3}\left(\frac{P(3)}{3}\right)^2}{2} = \frac{6!}{3!3!} \cdot \frac{(3!)^2}{3^2} \cdot \frac{1}{2} = \frac{6!}{18} = 40\]
\end{enumerate}

\section{Esercizio 4}
\begin{itemize}
    \item \(b_1 = 2\) \\
    0 \\
    1

    \item \(b_2 = 3\) \\
    00 \\
    01 \\
    10

    \item \(b_3 = 5\) \\
    000 \\
    001 \\
    010 \\
    100 \\
    101
\end{itemize}

Per individuare la formula si osserva che per ogni bit che si aggiunge, si ottengono tutte le stringhe di lunghezza \(n\) con uno zero aggiunto, più tuttle le stringhe che non terminano con un uno, esattamente \(b_{n - 1}\), con un uno aggiunto in fondo.
\[b_{n+1} = b_n + b_{n - 1}\]

\section{Esercizio 5}
Se \(n > m\) il numero di funzioni iniettive da \(n\) a \(m\) è 0. Altrimenti è uguale a
\[D(m, n) = \frac{m!}{n!(m - n)!}\]

\section{Esercizio 6}
\begin{enumerate}
    \item \[\sum_{i = 0}^n \left( D(m,i) \cdot \binom{n}{i} \right)\]
    \item \[\sum_{i = 0}^m \left( \binom{m}{i} \cdot D(n,i) \right)\]
\end{enumerate}

\end{document}