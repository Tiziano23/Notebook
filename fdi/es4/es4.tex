\documentclass{article}
\usepackage[utf8]{inputenc}
\usepackage[italian]{babel}
\usepackage[margin=1.25in]{geometry}
\usepackage{mathptmx}

\usepackage[dvipsnames]{xcolor}
\usepackage{listings}
\usepackage[outputdir=output]{minted}

\usepackage{hyperref}
\usepackage{graphicx}

\usepackage{amsfonts, amssymb, amsmath, amsthm}
\usepackage{stmaryrd}
\usepackage[thicklines]{cancel}

\hypersetup{
  colorlinks=true,
  linkcolor=black,
  filecolor=magenta,
  urlcolor=blue,
}

\definecolor{codekeyword}{HTML}{7cd2e2}
\definecolor{codestring}{HTML}{7df0d2}
\definecolor{codenumber}{HTML}{000000}
\definecolor{codecomment}{HTML}{7df0a1}
\definecolor{codebackground}{HTML}{1E1E1E}
\lstdefinelanguage{pseudolang}
{
  keywords={let, if, else, for, while, switch, case, return},
  string=[b]",
  comment=[l]//
}
\lstdefinestyle{pseudostyle}
{
  keywordstyle=\color{codekeyword},
  stringstyle=\color{codestring},
  numberstyle=\tiny\color{codenumber},
  commentstyle=\color{codecomment},
  backgroundcolor=\color{codebackground},
  basicstyle=\ttfamily\footnotesize\bfseries\color{White},
  breakatwhitespace=false,
  breaklines=true,
  captionpos=t,
  keepspaces=true,
  numbers=left,
  numbersep=5pt,
  showspaces=false,
  showstringspaces=false,
  showtabs=false,
  tabsize=2
}
\lstset{language=pseudolang, style=pseudostyle}

\newtheorem*{thm}{Teorema}
\theoremstyle{definition}
\newtheorem*{dfn}{Definizione}
\newtheorem*{oss}{Osservazione}
\newtheorem*{dms}{Dimostrazione}
\newtheorem*{eg}{Esempio}
\newtheorem*{nb}{N.B}

\newcommand{\quoted}[1]{``#1''}
\newcommand{\angled}[1]{\langle #1 \rangle}
\newcommand{\verted}[1]{\left\lvert #1 \right\rvert}
\newcommand{\bbracket}[1]{\llbracket #1 \rrbracket}

\title{Esercitazione 4}
\author{Tiziano Marzocchella [655205]}
\date{A.A. 2022/2023}

\begin{document}
\maketitle

\noindent Esercitazione svolta insieme a Francesco Faenza, Luca Cerretini e Emanuele Chilin

\section{Esercizio 1}

\begin{enumerate}
    \item \(mul: NTerm \times NTerm \rightarrow NTerm\):
          \begin{itemize}
              \item[\text{[C.B.]}] \(mul(x,Z) = Z\)
              \item[\text{[C.I.]}] \(mul(x, S(y)) = add(mul(x,y),x)\)
          \end{itemize}
          Dimostrazione per induzione della proprietà: \(val(mul(x,y)) = val(x) \cdot val(y)\)
          \begin{itemize}
              \item[\text{[C.B.]}] \(val(mul(x,z)) = val(x) \cdot val(Z)\)
              \begin{align*}
                  \{\text{C.B. di mul}\} &  & val(mul(x,z)) & = val(x) \cdot val(z) &  & \{\text{C.B. di val}\}         \\
                  \{\text{C.B. di val}\} &  & val(z)        & = val(x) \cdot 0      &  & \{\text{0 neutro di } \times\} \\
                                         &  & 0             & = 0                   &  &
              \end{align*}
              \item[\text{[P.I.]}] \(val(mul(x,y)) = val(x) \cdot val(y) \implies val(mul(x, S(y))) = val(x) \cdot val(S(y))\)
              \begin{align*}
                   & val(mul(x, S(y)))              &  & \{\text{C.I. mul}\}                 \\
                   & = val(add(mul(x,y),x))         &  & \{\text{Lemma valutazione di add}\} \\
                   & = val(mul(x,y)) + val(x)       &  & \{\text{Ipotesi induttiva}\}        \\
                   & = val(x) \cdot val(y) + val(x) &  & \{\text{Calcolo}\}                  \\
                   & = val(x) \cdot (val(y) + 1)    &  & \{\text{C.I. val}\}                 \\
                   & = val(x) \cdot (val(S(y)))
              \end{align*}
          \end{itemize}

    \item \(exp: NTerm \times NTerm \rightarrow NTerm\):
          \begin{itemize}
              \item[\text{[C.B.]}] \(exp(x,Z) = S(Z)\)
              \item[\text{[C.I.]}] \(exp(x, S(y)) = mul(x, exp(x,y))\)
          \end{itemize}
          Dimostrazione per induzione della proprietà: \(val(exp(x,y)) = val(x)^{val(y)}\)
          \begin{itemize}
              \item[\text{[C.B.]}] \(val(exp(x,Z)) = val(x)^{val(Z)}\)
              \begin{align*}
                  \{\text{C.B. di exp}\} &  & val(exp(x,Z)) & = val(x)^{val(Z)} &  & \{\text{C.B. di val}\}                 \\
                  \{\text{C.I. di val}\} &  & val(S(Z))     & = val(x)^0        &  & \{n^0 = 1\ \forall n \in \mathbb{N} \} \\
                  \{\text{C.B. di val}\} &  & val(Z) + 1    & = 1               &  &                                        \\
                                         &  & 1             & = 1               &  &
              \end{align*}
              \item[\text{[P.I.]}] \(val(exp(x,y)) = val(x)^{val(y)} \implies val(exp(x, S(y))) = val(x)^{val(S(y))}\)
              \begin{align*}
                   & val(exp(x, S(y)))              &  & \{\text{C.I. exp}\}                 \\
                   & = val(mul(x, exp(x,y)))        &  & \{\text{Lemma valutazione di mul}\} \\
                   & = val(x) \cdot val(exp(x,y))   &  & \{\text{Ipotesi induttiva}\}        \\
                   & = val(x) \cdot val(x)^{val(y)} &  & \{\text{Calcolo}\}                  \\
                   & = val(x)^{val(y) + 1}          &  & \{\text{C.I. val}\}                 \\
                   & = val(x)^{val(S(y))}
              \end{align*}
          \end{itemize}

    \item \begin{itemize}
              \item \(exp(0,0) \geq exp(0,1)\)
                    \begin{align*}
                        exp(0,0) & = val(exp(Z,Z)) = \\
                                 & = val(S(Z)) =     \\
                                 & = val(Z) + 1 =    \\
                                 & = 0 + 1 =         \\
                                 & = 1
                    \end{align*}
                    \begin{align*}
                        exp(0,1) & = val(exp(Z,S(Z))) =      \\
                                 & = val(mul(Z, exp(Z,Z))) = \\
                                 & = val(Z) * 1 =            \\
                                 & = 0 * 1 =                 \\
                                 & = 0
                    \end{align*}
                    \(1 \geq 0\)
              \item \(\lvert Fun(\varnothing, \varnothing) \rvert = val(exp(0,0)) = 1\)
              \item \(\lvert Fun(1, \varnothing) \rvert = val(exp(0,1)) = 0\)
          \end{itemize}
\end{enumerate}

\section{Esercizio 2}
\begin{enumerate}
    \item \(len(app(l_1, l_2)  = len(l_1) + len(l_2))\)
          \begin{itemize}
              \item[\text{[C.B.]}]
              \begin{align*}
                   & len(app([], l_2))    \\ \{\text{C.B. app}\}
                   & = len([]) + len(l_2)
              \end{align*}
              \item[\text{[C.I.]}] \(len(app(l_1,l_2)) = len(l_1) + len(l_2) \implies len(app(a:l_1, l_2)) = len(a:l_1) + len(l_2)\)
              \begin{align*}
                   & len(app(a: l_1, l_2)) =     \\
                   & = len(a: app(l_1, l_2)) =   \\
                   & = len(app(l_1,l_2)) + 1 =   \\
                   & = len(l_1) + len(l_2) + 1 = \\
                   & = len(a: l_1) + len(l_2)
              \end{align*}
          \end{itemize}
    \item \(len(rev(l)) = len(l)\)
          \begin{itemize}
              \item[\text{[C.B.]}]
              \begin{align*}
                   & len(rev([]])) \\ \{\text{C.B. rev}\}
                   & = len([])
              \end{align*}
              \item[\text{[C.I.]}] \(len(rev(l)) = len(l) \implies len(rev(a: l)) = len(a: l)\)
              \begin{align*}
                   & len(rev(a: l)) =          \\
                   & = len(app(rev(l), a:[]))  \\
                   & = len(rev(l)) + len(a:[]) \\
                   & = len(l) + len([]) + 1    \\
                   & = len(l) + 1              \\
                   & = len(a: l)
              \end{align*}
          \end{itemize}
\end{enumerate}

\section{Esercizio 3}
\begin{enumerate}
    \item \(max: BT_\mathbb{N} \rightarrow N \cup \{\infty\}\)
          \begin{enumerate}
              \item \(max(\lambda) = 0\)
              \item \(max(N(t_1, a, t_2)) = a, \forall a \in \mathbb{N} . \forall t_1,t_2 \in BT_\mathbb{N} .\ a \geq max(t_1) \land a \geq max(t_2)\)
              \item \(max(N(t_1, a, t_2)) = max(t_1), \forall a \in \mathbb{N} . \forall t_1,t_2 \in BT_\mathbb{N} .\ max(t_1) \geq max(t_2) \land max(t_1) \geq a\)
              \item \(max(N(t_1, a, t_2)) = max(t_2), \forall a \in \mathbb{N} . \forall t_1,t_2 \in BT_\mathbb{N} .\ max(t_2) \geq max(t_1) \land max(t_2) \geq a\)
          \end{enumerate}
    \item \(min: BT_\mathbb{N} \rightarrow N \cup \{\infty\}\)
          \begin{enumerate}
              \item \(min(\lambda) = \infty\)
              \item \(min(N(t_1, a, t_2)) = a, \forall a \in \mathbb{N} .\ \forall t_1,t_2 \in BT_\mathbb{N} .\ a \leq min(t_1) \land a \leq min(t_2)\)
              \item \(min(N(t_1, a, t_2)) = min(t_1), \forall a \in \mathbb{N} .\ \forall t_1,t_2 \in BT_\mathbb{N} .\ min(t_1) \leq min(t_2) \land min(t_1) \leq a\)
              \item \(min(N(t_1, a, t_2)) = min(t_2), \forall a \in \mathbb{N} .\ \forall t_1,t_2 \in BT_\mathbb{N} .\ min(t_2) \leq min(t_1) \land min(t_2) \leq a\)
          \end{enumerate}
\end{enumerate}

\section{Esercizio 5}

\begin{enumerate}
    \item \(ins: \mathbb{N} \times BT_\mathbb{N} \rightarrow BT_\mathbb{N}\)
          \begin{enumerate}
              \item \(ins(n, \lambda) = N(\lambda, n, \lambda)\)
              \item \(ins(n, N(t_1, a, t_2)) = N(ins(n, t_1), a, t_2), n < a\)
              \item \(ins(n, N(t_1, a, t_2)) = N(t_1, a, ins(n, t2)), \text{altrimenti}\)
          \end{enumerate}

    \item
          \begin{itemize}
              \item
                    \begin{align*}
                         & ins(4, N(N(\lambda,2,\lambda), 5, N(\lambda,7,\lambda))) =            \\
                         & = N(ins(4, N(\lambda, 2 \lambda)), 5, N(\lambda, 7, \lambda)) =       \\
                         & = N(N(\lambda, 2, ins(4, \lambda)), 5, N(\lambda, 7, \lambda)) =      \\
                         & = N(N(\lambda, 2, N(\lambda, 4, \lambda)), 5, N(\lambda, 7, \lambda))
                    \end{align*}
              \item
                    \begin{align*}
                         & ins(6, N(N(\lambda,2,\lambda), 5, N(\lambda,7,\lambda))) =            \\
                         & = N(N(\lambda, 2 \lambda), 5, ins(6, N(\lambda, 7, \lambda))) =       \\
                         & = N(N(\lambda, 2, \lambda), 5, N(ins(6, \lambda), 7, \lambda)) =      \\
                         & = N(N(\lambda, 2, \lambda), 5, N(N(\lambda, 6, \lambda), 7, \lambda))
                    \end{align*}
              \item
                    \begin{align*}
                         & ins(9, N(N(\lambda,2,\lambda), 5, N(\lambda,7,\lambda))) =            \\
                         & = N(N(\lambda, 2 \lambda), 5, ins(9, N(\lambda, 7, \lambda))) =       \\
                         & = N(N(\lambda, 2 \lambda), 5, N(\lambda, 7, ins(9,\lambda))) =        \\
                         & = N(N(\lambda, 2 \lambda), 5, N(\lambda, 7, N(\lambda, 9,\lambda))) = \\
                    \end{align*}
          \end{itemize}
\end{enumerate}

\section{Esercizio 6}
\(ordinato(t) \implies ordinato(ins(n,t))\)

\begin{itemize}
    \item \(ordinato(\lambda) \implies ordinato(ins(n, \lambda))\)
          \begin{align*}
               & ordinato(ins(n, \lambda))                   \\
               & = ordinato(N(\lambda, n, \lambda))          \\
               & = ordinato(\lambda) \land ordinato(\lambda) \\
               & = true \land true
          \end{align*}

    \item \(\forall a \in \mathbb{N} . \forall t_1,t_2 \in BT_\mathbb{N} . (P(t_1) \land P(t_2) \implies P(N(t_1,a,t_2)))\)
          \begin{align*}
                       & (ordinato(t_1) \implies ordinato(ins(n,t_1)) \land ordinato(t_2) \implies ordinato(ins(n, t_2))) \\
              \implies & (ordinato(N(t_1,n,t_2)) \implies ordinato(ins(n, N(t_1, a, t_2))))
          \end{align*}

          \(ordinato(N(t_1,nt_2)) \implies ordinato(ins(n, N(t_1, a, t_2)))\)
\end{itemize}

\end{document}