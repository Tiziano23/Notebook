\documentclass{article}
\usepackage[utf8]{inputenc}
\usepackage[italian]{babel}
\usepackage[margin=1.25in]{geometry}

\usepackage{amsfonts, amssymb}
\usepackage{amsmath, amsthm}
\usepackage[thicklines]{cancel}

\usepackage{graphicx}
\usepackage{hyperref}

\hypersetup{
colorlinks=true,
linkcolor=black,
filecolor=magenta,
urlcolor=blue,
}

\newtheorem*{thm}{Teorema}

\theoremstyle{definition}
\newtheorem*{dfn}{Definizione}
\newtheorem*{oss}{Osservazione}
\newtheorem*{dms}{Dimostrazione}
\newtheorem*{eg}{Esempio}
\newtheorem*{nb}{N.B}

\newcommand{\quoted}[1]{``#1''}
\newcommand{\angled}[1]{\langle #1 \rangle}

\title{Fondamenti dell'Informatica}
\author{Tiziano Marzocchella}
\date{A.A. 2022/2023}

\begin{document}
\maketitle
\tableofcontents
\newpage

\section{Introduzione}
In questo corso affronteremo...
\begin{itemize}
    \item Linguaggio matematico di base: insiemi, relazioni, funzioni, ecc...
    \item Strutture discrete: coppie, n-uple, stringhe, grafi, alberi, ecc..
    \item Calcolo combinatorio: permutazioni, disposizioni, combinazioni
    \item Tecniche di dimostrazione: simboliche, discorsive, per induzione
    \item Specifiche formali: linguaggi formali, logica, ricorsione
\end{itemize}

%Modalità di esame

%Materiale didattico

\pagebreak

\section{Insiemi}
Un \emph{insieme} è una collezione di oggetti, detti elementi.

\subsection{Notazione}
\begin{itemize}
    \item Utilizziamo \(A, B, C, \dots\) per denotare insiemi generici o direttamente un nome specifico
    \item Utilizziamo \(a, b, c, \dots\) per denotare gli elementi di un insieme
    \item \(a \in A\), a \emph{appartiene a} A
    \item \(a \not\in A\), a \emph{non appartiene a} A
\end{itemize}

\subsection{Rappresentazione}
\begin{itemize}
    \item \emph{Estensionale}, per enumerazione, ovvero elencare in una lista tutti gli elementi
    \item \emph{Intenzionale}, per proprietà
\end{itemize}

Nella rappresentazione intenzionale, un insieme \(X\) contiene tutti gli elementi \(x\) dell'insieme di riferimento \(A\) che soddisfano la proprietà \(P\).
\[X = \{ x \in A \mid P(x) \}\]

\begin{eg}
    \begin{align*}
        \mathbb{N}^p & = \{n \mid n \in N \text{ e n è divisibile per 2} \}         \\
        \mathbb{Q}   & = \{\frac{n}{m} \mid n \in \mathbb{Z}, m \in \mathbb{N}^+ \}
    \end{align*}
    \[\]
\end{eg}

\subsection{Confronto di insiemi}
\begin{dfn}
    Due insiemi sono uguali \((A = B)\) se hanno gli stessi elementi, altrimenti sono diversi \((A \not= B)\)
\end{dfn}

\begin{eg}
    \begin{align*}
        V = \{a,e,i,o,u\} & V1 = \{a,e,i,o,e,a\}
    \end{align*}
\end{eg}

\subsection{Inclusione}
\begin{dfn}
    \(A\) è un sottoinsieme di \(B\) \((A \subseteq B)\) se ogni elemento di \(A\) appartiene anche a \(B\).
    \(A\) è un sottoinsieme proprio di \(B\) se  \((A \subseteq B) \text{ e } A \neq B\)
\end{dfn}

\subsection{Proprietà di uguaglianza e inclusione}
\begin{itemize}
    \item Riflessività, vale sempre per un insieme e se stesso
          \[A = A \quad A \subseteq A\]
    \item Transitività
          \[\text{Se } A = B \text{ e } B = C \text{ allora } A = C\]
          \[\text{Se } A \subseteq B \text{ e } B \subseteq C \text{ allora } A \subseteq C\]
    \item Simmetria
          \[\text{Se } A = B \text{ allora } B = A\]
    \item Anti-simmetria
          \[\text{Se } A \subseteq B \text{ e } B \subseteq A \text{ allora } A = B\]
\end{itemize}

\subsection{Operazioni su insiemi}
\begin{itemize}
    \item Unione
    \item Intersezione
    \item Differenza
    \item Complemento
\end{itemize}

\subsection{Uguaglianze e leggi}
Una uguaglianza valida, cioè vera per tutti i possibili casi, si chiama \emph{legge}.

\subsection{Operatori booleani}
Sono operazioni su valori booleani (vero o falso) che possiamo utilizzare per scrivere espressioni complesse che denotano insiemi.

Per dimostrare la validità di una uguaglianza si utilizzano dimostrazioni di tipo:
\begin{itemize}
    \item discorsive
    \item grafiche
    \item per sostituzione
\end{itemize}

\subsection{Leggi per operatori su insiemi}
\begin{itemize}
    \item Leggi per \textbf{unione} e \textbf{intersezione}
          \begin{itemize}
              \item associatività
              \item unità
              \item commutatività
              \item idempotenza
              \item assorbimento
          \end{itemize}
    \item Leggi che collegano \(\cup\), \(\cap\) e \(\overline{()}\)
    \item Leggi per \(\setminus\)
    \item Complemento
    \item Convoluzione
    \item De Morgan
    \item Universo e insieme vuoto
\end{itemize}

\subsection{Cardinalità}
La cardinalità di un insieme finito è il numero di elementi al suo interno.
\begin{itemize}
    \item \(\lvert \varnothing \rvert = 0\)
    \item \(A \subseteq B \implies \lvert A \rvert \leq \lvert B \rvert\)
    \item \(A = B \implies \lvert A \rvert = \lvert B\rvert \)
    \item \(A \subset B \implies \lvert A \rvert < \lvert B \rvert\), non vale per insiemi infiniti
\end{itemize}

\subsection{Insiemi di insiemi}

\subsection{Numeri naturali come insiemi}

\subsection{Insieme delle parti}
Dato un insieme \(A\), l'insieme delle parti di \(A\) è:
\[\mathcal{P}(A) = \{X \mid X \subseteq A\}\]
\begin{itemize}
    \item \(\mathcal{P}(\{0,1,2\}) = \{\varnothing, \{0\}, \{1\}, \{2\}, \{0,1\}, \{0,2\}, \{1,2\}, \{0,1,2\}\}\)
    \item \(\mathcal{P}(\varnothing) = \{\varnothing\}\)
    \item \(\lvert\mathcal{P}(A)\rvert = 2^{\lvert A \rvert}\)
\end{itemize}

\subsection{Prodotto cartesiano}
Il prodotto cartesiano di due insiemi è l'insieme formato da tutte le coppie ordinate \((a,b)\) tali che \(a \in A\) e \(b \in B\).
\[A \times B = \{(a,b) \mid a \in A, b \in B\}\]

\subsection{Famiglie di insiemi}
Sia \(I\) un insieme finito (di indici).\\
Una famiglia di insiemi \(F\) indicizzata da \(I\) è:
\[F = \{A_i \mid i \in I\} = \{A_i\}_{i \in I}\]

\begin{itemize}
    \item \(\cup F = \cup_{i \in I}A_i\)
    \item \(\cap F = \cap_{i \in I}A_i\)
\end{itemize}

\subsection{Partizioni}
Una partizione su un insieme \(A\) è una famiglia di sottoinsiemi di \(A\) tale che:
\begin{itemize}
    \item ogni insieme \(A_i\) è diverso da \(\varnothing\)
    \item \(\cup_{i \in I}A_i = A\) (copertura di A)
    \item Presi \(i \neq j\) si ha: \(A_i \cap A_j = \varnothing\)
\end{itemize}
Serve a classificare \quoted{esclusivamente} gli elementi di un insieme.

\subsection{Paradosso di Russel}

\pagebreak

\section{Relazioni}
\subsection{Nozioni di base}
Dati gli insieme \(A\) e \(B\), una relazione \(R \subseteq A \times B\) si scrive
\[R \in Rel(A,B)\]
\[R: A \leftrightarrow B\]
dove
\[Rel(A,B) = \mathcal{P}(A \times B)\]

\begin{itemize}
    \item relazione vuota: \(\varnothing \subseteq A \times B\) oppure \(\varnothing_{A,B}\)
    \item relazione completa: \(A \times B\)
\end{itemize}

\subsubsection{Relazioni su un singolo insieme}
Chiamiamo le relazioni con insieme di partenza e arrivo coincidenti \textbf{Relazioni su \(A\)}
\begin{eg}
    \[Succ = \{(x,y) \in \mathbb{N} \in \mathbb{N} \mid y = x + 1\} \in Rel(\mathbb{N}, \mathbb{N}) \quad \text{Relazione su } \mathbb{N}\]
\end{eg}

\subsubsection{Relazione identità}
Per un insieme \(A\) la relazione identità è:
\[Id_A = \{(x,x) \mid x \in A\}\]
una relazione che associa ogni elemento di \(A\) con se stesso. Viene definita solo per relazioni su singoli insiemi

\subsection{Operazioni su relazioni}
Possiamo definire nuove relazioni usando vari operatori insiemistici.\\
Date due relazioni \(R\) e \(S\) su \(A\) e \(B\),
\[R, S \in Rel(A,B)\]
\begin{itemize}
    \item \(R \cup S \subseteq A \times B\), unione tra \(R\) e \(S\)
    \item \(R \cap S \subseteq A \times B\), intersezione tra \(R\) e \(S\)
    \item \(R \setminus S \subseteq A \times B\), tutti gli elementi di \(R\) tolti gli elementi di \(S\)
    \item \(\overline{R} = (A \times B \setminus R) \subseteq A \times B\), complemento di \(R\). Scelgo \(A \times B\) come universo per garantire l'inclusione
\end{itemize}
Valgono tutte le leggi che abbiamo visto per gli insiemi, ma con \(A \times B\) preso come insieme universo.

\subsection{Composizione di relazioni}
Siano \(R: A \leftrightarrow B\) e \(S: B \leftrightarrow C\), la \textbf{composizione} di \(R\) con \(S\) è:
\[R;S: A \leftrightarrow C\]
\[R;S = \{(x,y) \in A \times C \mid (\exists y \in B . (x,y) \in R \land (y,z) \in S\}\]
\(R;S\) contiene tutti le coppie per le quali c'è un \quoted{cammino} da \(A\) a \(C\), seguendo le frecce delle relazioni.

\subsubsection{Leggi per la composizione}
Per tutti gli insiemi \(A,B,C,D\) e tutte le relazioni \(R: A \leftrightarrow B\), \(S: B \leftrightarrow C\) e \(T: C \leftrightarrow D\) valgono le seguenti leggi
\begin{itemize}
    \item Associatività: \(R;(S;T) = (R;S);T\)
    \item Unità: \(Id_A;R = R = R;Id_B\)
    \item Assorbimento: \(R;\varnothing_{B,C} = \varnothing_{A,C} = \varnothing_{A,B};S\)
    \item Distributività
\end{itemize}

\subsection{Relazione opposta}
Sia \(R: A \leftrightarrow B\) una relazione. La sua relazione \textbf{opposta} è:
\[R^{op} = \{(y,x) \in B \times A \mid (x,y) \in R\}\]

\subsubsection{Leggi per la relazione opposta}
Per tutti gli insiemi \(A,B,C\) e tutte le relazioni \(R: A \leftrightarrow B\), \(S: B \leftrightarrow C\) valgono le seguenti leggi
\begin{itemize}
    \item Composizione: \(R^{op};S^{op} = (R;S)^{op}\)
    \item Distributività: \(\)
\end{itemize}

\subsection{Proprietà delle relazioni}
\subsubsection{Totalità}
\subsubsection{Univalenza}
\subsubsection{Suriettività}
\subsubsection{Iniettività}
\subsubsection{Osservazioni}
\textbf{Schema TUSI}

\noindent\textbf{Dualità}
\begin{itemize}
    \item \(R \text{ è totale}     \iff R^{op} \text{è suriettiva}\)
    \item \(R \text{ è univalente} \iff R^{op} \text{è iniettiva}\)
    \item \(R \text{ è suriettiva} \iff R^{op} \text{è totale}\)
    \item \(R \text{ è iniettiva}  \iff R^{op} \text{è univalente}\)
\end{itemize}
\noindent\textbf{Caratterizzazione}
\begin{align*}
     & R \text{ è totale} \iff Id_A \subseteq R;R^{op}     &  &  & R \text{ è suriettiva} \iff Id_B \subseteq R^{op};R \\
     & R \text{ è univalente} \iff R;R^{op} \subseteq Id_A &  &  & R \text{ è iniettiva} \iff R^{op};R \subseteq Id_B
\end{align*}

\pagebreak

\section{Funzioni}
Dati due insiemi \(A\) e \(B\), \(R: A \leftrightarrow B\) è una funzione se
\[\forall a \in A (\exists \text{ esattamente un } b \in B . (a,b) \in R\]
ovvero se \(R\) è \textbf{totale} e \textbf{univalente}. Vediamo alcuni esempi:
\begin{itemize}
    \item \(Id_A\) è una funzione
    \item \(A \times B\) non è una funzione, ma
          \begin{itemize}
              \item se \(\lvert B \rvert = 1\), è una funzione
              \item se \(\lvert A \rvert = 0\) e \(\lvert B \rvert = 0\), è una funzione
          \end{itemize}
    \item \(\varnothing: A \leftrightarrow B\), non è una funzione, ma
          \begin{itemize}
              \item se \(\lvert A \rvert = 0\), è una funzione
          \end{itemize}
\end{itemize}

\subsection{Composizione di funzioni}
Per tutti gli insiemi \(A\), \(B\) e \(C\) e tutte le funzioni \(f: A \rightarrow B\) e \(g: B \rightarrow C\), la relazione \(f;g\) è una funzione
\[f;g: A \rightarrow C\]
\(g \circ f\) o \(gf\), notazione matematica

\subsection{Biiezioni}
Dati \(A\) e \(B\), la relazione \(R: A \leftrightarrow B\) è una biiezione se è totale, univalente, iniettiva e surgettiva.
\[\forall a \in A \text{ esiste esattamente un } b \in B . (a,b) \in R \land \forall b \in B \text{ esiste esattamente un } a \in A . (a,b) \in R\]
Il numero di biiezioni tra due insiemi \(A\) e \(B\) con \(\lvert A \rvert = \lvert B \rvert = n\) è uguale a
\[n!\]

\subsubsection{Osservazioni}
Per tutti gli \(A,B,C\) e \(i: A \rightarrow B\) e \(j: B \rightarrow C\) vale che
\begin{itemize}
    \item \(Id_A\) è una biiezione
    \item \(i;j\) è una biiezione
    \item \(i^{op}\) è una biiezione
\end{itemize}

\subsubsection{Teorema di caratterizzazione}
Per tutti gli insiemi \(A\),\(B\), la relazione \(R: A \leftrightarrow B\) è biiezione se e solo se:
\[Id_A = R;R^{op} \text{ e } Id_B = R^{op};R\]
\[R \text{ biiezione} \iff Id_A \subseteq R;R^{op} \land R;R^{op} \subseteq Id_A \land Id_B \subseteq R^{op};R \land R^{op};R \subseteq Id_B \]

\subsubsection{Insiemi in biiezioni}
\[A \cong B\]
Due insiemi sono in biiezione se esiste una biiezione \(i: A \rightarrow B\)

\noindent\textbf{Proprietà della biiezione}
Dati gli insiemi \(A,B,C\) vale:
\begin{itemize}
    \item Riflessività, \(A \cong A\)
    \item Simmetria, \(A \cong B \implies B \cong A\)
    \item Transitività, \(A \cong A \land B \cong C \implies A \cong C\)
\end{itemize}

\pagebreak

\section{Induzione}
L'induzione ci permette di definire insiemi, anche infiniti.

\begin{eg}
    Se prendiamo la definizione dell'insieme dei naturali
    \[\mathbb{N} = \{0,1,2,\ldots\}\]
    possiamo dire che non è una definizione soddisfacente perché
    \begin{itemize}
        \item cambiando l'ordine degli elementi nell'insieme la notazione con i puntini sospensivi non ha più senso. \[\mathbb{N} = \{2,0,1,\ldots\}\]
        \item Un programma non saprebbe determinare il significato dei puntini sospensivi.
    \end{itemize}
\end{eg}

La definizione per induzione si può applicare sia agli insiemi che alle funzioni.

\subsection{Definizione induttiva di insieme}
Per definire un insieme \(A\) con la definizione induttiva occorre definire 3 clausole:
\begin{enumerate}
    \item Clausola \textbf{base}, che esplicita gli elementi di \(A\)
    \item Clausola \textbf{induttiva}, che definisce come utilizzare gli elementi già presenti in A per costruire gli altri elementi
    \item Clausola \textbf{terminale}, che stabilisce quando l'insieme \(A\) non contiene altri elementi, ovvero è il più piccolo insieme che soddisfa le clausole 1 e 2.
\end{enumerate}

\subsection{Definizione induttiva di funzione}
Una funzione \(f: A \rightarrow B\) definita per induzione segue le seguenti clausole
\begin{enumerate}
    \item Clausola \textbf{base}, ovvero il valore di \(f(a)\) per ogni \(a \in A\) (secondo la clausola base della definizione di A)
    \item Clausola \textbf{induttiva}, ovvero le regole utilizzate per calcolare \(f(a)\) usando i valori di \(f\) per elementi che sono già in \(A\)
\end{enumerate}
La definizione induttiva garantisce una funzione totale

\subsection{Dimostrazione per induzione}
Si ha una proprietà \(P\) sui naturali
\[P: \mathbb{N} \rightarrow Bool\]
Verifichiamo le seguenti affermazioni
\begin{itemize}
    \item Caso base: \(P(0)\) è vera
    \item Passo induttivo: Per ogni \(m \in \mathbb{N}\)
          \[\left(P(n) \implies P(n + 1)\right)\]
\end{itemize}
Se entrambe sono vere allora \(P(m)\) è vera per ogni \(m \in \mathbb{N}\)

\subsubsection{Regola di inferenza}
Possiamo esprimere la dimostrazione per induzione attraverso la notazione di regola di inferenza
\[\frac{P(0), \forall n \in \mathbb{N} . \left(P(n) \implies P(n + 1)\right)}{\forall m \in \mathbb{N} . P(m)}\]
che si legge: \quoted{Per dimostrare ciò che sta sotto la riga, basta dimostrare ciò che sta sopra la riga}

\pagebreak

\section{Grafi}
I grafi ci permettono di modellare precisamente e in modo visualmente intuitivo relazioni tra elementi di un insieme.

Esistono grafi \emph{orientati} e \emph{non orientati}. I grafi orientati sono più direttamente collegati al concetto di relazione.

\subsection{Grafo orientato}
Un \emph{grafo orientato} è una relazione \(E: V \leftrightarrow V\) su un insieme \textbf{finito} \(V\). Scriviamo: \(G = (E,V)\)
\begin{itemize}
    \item \(V\): insieme di \textbf{nodi} o \textbf{vertici}
    \item \(E\): insieme di \textbf{archi} o \textbf{lati} (coppie)
\end{itemize}

Questi grafi sono \emph{orientati} perchè ad un arco dal nodo \(x\) al nodo \(y\) possiamo assegnare un verso, e quindi possiamo anche creare archi che partono e arrivano nello stesso nodo, chiamati \textbf{cappio} o \textbf{loop}. Ne risulta che presi due nodi distinti \(x\) e \(y\), un arco \((x,y) \in E \neq (y,x) \in E\).

Dato un grafo \(G\) indicheremo:
\begin{itemize}
    \item con \(n\) il numero dei nodi del grafo, cioè \(n = \lvert V\rvert\)
    \item con \(m\) il numero di archi, ovvero \(m = \lvert E\rvert\)
\end{itemize}

\subsection{Vicinato e grado dei nodi}
Due nodi x e y sono adiacenti se c'è un arco da x a y o c'è un'arco da y a x
\[(x,y) \in E \lor (y,x) \in E\]
Per ogni nodo possiamo indicare
\begin{itemize}
    \item Vicinato in uscita (stella uscente) di x 
    \[N^+(x) = \{y \mid (x,y) \in E\}\]
    \item Vicinato in ingresso (stella entrante) di x 
    \[N^-(x) = \{y \mid (y,x) \in E\}\]
    \item Grado in uscita di x:
    \[d^+_x = \lvert N^+(x) \rvert\]
    \item Grado in ingresso di x:
    \[d^-_x = \lvert N^-(x) \rvert\]
\end{itemize}

\noindent\textbf{Hand-Shaking Lemma per grafi orientati}
\[\sum_{x \in V} d_x^- = \sum_{x \in V} d_x^+ = \lvert E \rvert\]

\subsection{Grafo come relazione}
Essendo un grafo una relazione su un insieme finito di nodi, ovviamente possiamo verificare le proprietà TUSI.
Le proprietà TUSI (Totale, Univalente, Surgettiva, Iniettiva) diventano condizioni sul grado dei nodi.
\begin{enumerate}
    \item \(E: V \leftrightarrow V\) è \textbf{totale} se e solo se per ogni nodo \(x \in V\) vale \(d_x^+ \geq 1\);
    \item \(E: V \leftrightarrow V\) è \textbf{univalente} se e solo se per ogni nodo \(x \in V\) vale \(d_x^+ \leq 1\);
    \item \(E: V \leftrightarrow V\) è \textbf{surgettiva} se e solo se per ogni nodo \(x \in V\) vale \(d_x^- \geq 1\);
    \item \(E: V \leftrightarrow V\) è \textbf{iniettiva} se e solo se per ogni nodo \(x \in V\) vale \(d_x^- \leq 1\);
\end{enumerate}

\subsection{Cammini}
\begin{itemize}
    \item Un \emph{walk} è una sequenza di nodi \(P = v_0, \ldots, v_k\) con \(k \in \mathbb{N}\) tale che \((v_i,v_{i+1}) \in E\) per ogni \(i \in \{1,\ldots,k\}\).
    \item Un \emph{trail} è un \emph{walk} che non attraversa più volte lo stesso \textbf{arco}.
    \item Un \emph{path} è un \emph{trail} che non attraversa più volte lo stesso \textbf{nodo}.
\end{itemize}

La lunghezza di un walk è \(k\) e gli estremi sono \(v_0\) e \(v_k\).

Un walk di lunghezza zero è costituito solo dal nodo \(v_0\).

Esiste un walk di lunghezza \(n \in \mathbb{N}\) da \(x\) a \(y\) se e solo se \((x,y) \in E^n\).

\begin{dms}
    Dimostriamo per induzione questa proprietà
    \begin{itemize}
        \item \textbf{Caso base}: Sia \(v \in V\), v è walk di lunghezza 0 per definizione v è un walk da v a v, ma \(E^0 = Id_v\), quindi \((v,v) \in E^0 = Id_v\).
        \item \textbf{Passo induttivo}: Esiste \(v_1 v_2, \ldots, v_{n+1}\) un walk di lunghezza n + 1 se e solo se c'è un arco \((v_0,v_1) \in E\) e un walk \(v_0, v_1, \ldots, v_{n+1}\) di lunghezza n.
    \end{itemize}
\end{dms}

Se esiste un walk da \(x\) a \(y\) allora esiste un trail da \(x\) a \(y\).

\subsection{Walk chiusi, circuiti e cicli}
Un walk è detto chiuso se 

\subsection{Connettività}
Un grafo orientato \(G = (V,E)\) è fortemente connesso se per ogni coppia di nodi \(x,y \in V\) esiste un walk da \(x\) a \(y\).

Una componente fortemente connessa di G è un sotto insieme non vuoto di nodi \(U \subseteq V\) tale che
\begin{enumerate}
    \item Per ogni copppia di nodi \(x, y \in U\) esiste un walk da \(x\) a \(y\).
    \item Se \(U' \subseteq V\) soddisfa la 1. e \(U \subseteq U'\), allora \(U = U'\).
\end{enumerate}

\textbf{Altre proprietà}
\begin{itemize}
    \item \(G = (V,E)\) è fortemente connessa se e solo se \(V \times V \subseteq E^*\)
    \item \((x,y) \in E^* \cap (E^*)^{op}\) se e solo se \(x\) e \(y\) appartengono alla stessa SCC.
    \item \(x\) e \(y\) appartengono alla stessa SCC se e solo se esiste un walk chiuso che li attraversa entrambi.
\end{itemize}

\subsection{Directed Acyclic Graphs}
Un grafo orientato senza cicli si chiama \textbf{Directed Acyclic Graph} (DAG)

I nodi con grado di ingresso 0 vengono chiamati \emph{sorgenti} e i nodi con grado di uscita 0 si chiamano \emph{pozzi}.

Se \(G = (V,E)\) è un DAG, allora \(E^*\) è un ordinamento parziale.

\begin{dfn}{Ordinamenti topologici:}
    Dato un DAG, un ordinamento topologico di G è una biiezione \(\eta: V \rightarrow n = \{0,1,2,\ldots,n-1\}\) tale che
    \begin{center}
        Per ogni arco \((x,y) \in E\) vale che \(\eta(x) < \eta(y)\)
    \end{center}
\end{dfn}

\subsection{Cammini e cicli Euleriani}

\begin{dfn}{Circuito euleriano: } 
    Un circuito euleriano è un circuito che passa esattamente una volta per tutti gli archi del grafo.
\end{dfn}

\begin{dfn}{Trail euleriano: }
    Un percorso euleriano è un trail che passa esattamente una volta per tutti gli archi del grafo.
\end{dfn}

\vspace{1 cm}

Eulero ha sviluppato un teorema per decidere se esiste o meno un circuito o trail che passa per tutti gli archi.

\begin{dfn}{Teorema di Eulero}\\
    Dato un grafo non orientato e connesso,
    \begin{enumerate}
        \item Eiste un circuito euleriano se e solo se tutti i nodi hanno grado pari.
        \item Esiste un trail euleriano da \(x\) a \(y\) se e solo se \(x\) e \(y\) sono gli unici nodi di grado dispari
    \end{enumerate}
\end{dfn}

\subsection{Cicli e path hamiltoniani}
Dato un grafo connesso (orientato o no orientato), un ciclo hamiltoniano è un ciclo che passa esattamente una volta per tutti i nodi del grafo.

Un path hamiltoniano può partire e arrivare in nodi distinti.


\section{Alberi}
Un albero è un grafo non orientato, connesso, aciclico e non vuoto.

\subsection{Terminologia}
\begin{itemize}
    \item Una \textbf{foresta} è un grafo non orientato, aciclico e non vuoto.
    \item Una \textbf{foglia} è un nodo con grado 1.
    \item Un \textbf{nodo interno} è un nodo con grado > 1
    \item Un \textbf{albero radicato} è un albero con un \quoted{nodo speciale} ovvero la \textbf{radice}
\end{itemize}
\end{document}