\documentclass{article}
\usepackage[utf8]{inputenc}
\usepackage[italian]{babel}
\usepackage[margin=1.25in]{geometry}

\usepackage{amsfonts, amssymb}
\usepackage{amsmath, amsthm}
\usepackage[thicklines]{cancel}

\usepackage{graphicx}
\usepackage{hyperref}

\hypersetup{
colorlinks=true,
linkcolor=black,
filecolor=magenta,
urlcolor=blue,
}

\newtheorem*{thm}{Teorema}

\theoremstyle{definition}
\newtheorem*{dfn}{Definizione}
\newtheorem*{oss}{Osservazione}
\newtheorem*{dms}{Dimostrazione}
\newtheorem*{eg}{Esempio}
\newtheorem*{nb}{N.B}

\newcommand{\quoted}[1]{``#1''}
\newcommand{\angled}[1]{\langle #1 \rangle}

\title{Esercitazione 1}
\author{Tiziano Marzocchella [#-Matricola: 655205]}
\date{A.A. 2022/2023}

\begin{document}
\section{Esercizio 1}
\begin{enumerate}
    \item Dimostrazione per induzione della proprietà \(P: \forall n \in \mathbb{N} . f(n) < 1\) \\
          \textbf{Caso base}, \(P(0)\) è vera:
          \begin{align*}
              P(0) & \implies f(0) < 1        \\
                   & \implies \frac{1}{2} < 1
          \end{align*}

          \textbf{Passo induttivo}, \(P(n) \implies P(n + 1)\):
          \[f(n) < 1 \implies f(n + 1) < 1\]
          \begin{align*}
              (f(n + 1) < 1) \equiv \quad \text{\{Clausola induttiva di f\}} \\
              (\frac{1}{2 - f(n)} < 1) \equiv \quad \text{\{Calcolo\}}  \\
              (\frac{1}{2 - f(n)} < \frac{1}{2 - 1}) \equiv \quad \text{\{Per osservazione(1) e ipotesi induttiva, \(f(n) < 1 < 2\)\}}
          \end{align*}

    \item Dimostrazione per induzione della proprietà \(P: \forall n \in \mathbb{N} . f(n) \geq \frac{1}{2}\) \\
          \textbf{Caso base}, \(P(0)\) è vera:
          \begin{align*}
              P(0) & \implies f(0) \geq \frac{1}{2} \\
                   & \implies \frac{1}{2} \geq \frac{1}{2}
          \end{align*}

          \textbf{Passo induttivo}, \(P(n) \implies P(n + 1)\):
          \[f(n) < 1 \implies f(n + 1) < 1\]
          \begin{align*}
              \left(f(n + 1) \geq \frac{1}{2}\right)               \equiv & \quad \text{\{Clausola induttiva di f\}} \\
              \left(\frac{1}{2 - f(n)} \geq \frac{1}{2}\right)     \equiv & \quad \text{\{Calcolo\}} \\
              \left(\frac{1}{2 - f(n)} \geq \frac{1}{2 - 0}\right) \equiv & \quad \text{\{Per osservazione(1) e ipotesi induttiva, \(0 < f(n) < 2\)\}} \\
              \left(\right)
          \end{align*}
\end{enumerate}

\section{Esercizio 2}
\begin{align*}
    R \in Rel(A,A) \\
    R^{op} \cap Id_A &= (R \cap Id_A)^{op} && \{id-op\} \\
    R^{op} \cap Id_A^{op} &= (R \cap Id_A)^{op} && \{distributività di .^{op} su \cap\} \\
    (R \cap Id_A)^{op} &= (R \cap Id_A)^{op}
\end{align*}

\section{Esercizio 3}
\(A = \{a,b\} \quad R = \{(a,b) \quad R;R = \varnothing\}\)

\section{Esercizio 4}
\
\begin{align*}
    R \in Rel(A,A) \\
    R^{op} \cap Id_A &= (R \cap Id_A)^{op} && \{id-op\} \\
    R^{op} \cap Id_A^{op} &= (R \cap Id_A)^{op} && \{distributività di .^{op} su \cap\} \\
    (R \cap Id_A)^{op} &= (R \cap Id_A)^{op}
\end{align*}
\end{document}