\documentclass{article}
\usepackage[utf8]{inputenc}
\usepackage[italian]{babel}
\usepackage[margin=1.25in]{geometry}

\usepackage{amsfonts, amssymb}
\usepackage{amsmath, amsthm}
\usepackage[thicklines]{cancel}

\usepackage{graphicx}
\usepackage{hyperref}

\hypersetup{
colorlinks=true,
linkcolor=black,
filecolor=magenta,
urlcolor=blue,
}

\newtheorem*{thm}{Teorema}

\theoremstyle{definition}
\newtheorem*{dfn}{Definizione}
\newtheorem*{oss}{Osservazione}
\newtheorem*{dms}{Dimostrazione}
\newtheorem*{eg}{Esempio}
\newtheorem*{nb}{N.B}

\newcommand{\quoted}[1]{``#1''}
\newcommand{\angled}[1]{\langle #1 \rangle}
\newcommand{\numN}{\mathbb{N}}
\newcommand{\numZ}{\mathbb{Z}}
\newcommand{\numQ}{\mathbb{Q}}
\newcommand{\numR}{\mathbb{R}}

\newcommand{\gap}{\hspace{.4cm}}
\newcommand{\ggap}{\hspace{.6cm}}
\newcommand{\Gap}{\hspace{1cm}}
\newcommand{\GAP}{\hspace{2.5cm}}

%aliases
\newcommand{\funcs}[3]{#1: #2 \rightarrow #3}

\title{Analisi Matematica}
\author{Tiziano Marzocchella}
\date{A.A. 2022/2023}

\begin{document}
\maketitle
\tableofcontents
\newpage

\section{Introduzione - Nozioni di base}
\subsection{Insiemi numerici}
\begin{itemize}
    \item \(\numN\) - Numeri naturali (interi positivi)
    \item \(\numZ\) - Numeri interi (anche i negativi)
    \item \(\numQ\) - Numeri razionali (frazioni), tutti i numeri \(\frac{p}{q} \text{ con } p, q \in \numZ \text{ e } q \neq 0\)
    \item \(\numR\) - Numeri reali, tutti i numeri compresi anche quelli \(\notin \numQ\)
    \begin{eg}
    \(\sqrt{2} \notin \numQ \hspace{2cm} \pi \notin \mathbb{Q} \hspace{2cm} e \notin \numQ\)
    \end{eg}
\end{itemize}

\subsection{Retta reale}
I numeri reali sono \emph{oggetti} su cui possiamo svolgere operazioni proprio come sui naturali.
\[\numN \subset \numZ \subset \numQ \subset \numR\]
In questo corso consideriamo \(\subset\) e \(\subseteq\) equivalenti, quindi
\[A \subset B \land B \subset A \implies A=B\]
Per denotare un insieme \(A \subset B\) ma \(A \neq B\) si usa
\[A \subsetneqq B\]

\subsection{Intervalli di \(\numR\)}
\begin{dfn}
\(I \subset \numR\) si dice intervallo di \(\numR\) se \(\forall x, y \in I\) con \(x < y\)\\
dato \(z \in \numR\) tale che \(x < z < y\) risulta che \(z \in I\)
\end{dfn}

\begin{eg}
\(A = \{ x \in \numR : x \neq 0\}\)
%drawing
\begin{align*}
    -2 \in A, 3 \in A && -2 < 0 < 3,\gap 0 \notin A 
\end{align*}
\underline{\(A\) \textbf{non} è un intervallo.}
\end{eg}

\begin{eg}
\(B = \{ x \in \numR : \lvert x \rvert > 3\}\)
%drawing
\[z \notin B, \gap x < z < y\]
\underline{\(B\) \textbf{non} è un intervallo.}
\end{eg}

\noindent\vspace{.2cm}\large\textbf{Notazione}\\
Dati \(a,b \in \numR\), con \(a < b\) scriviamo:
\begin{itemize}
    \item \([a, b] = \{x \in \numR : a \leq x \leq b\}\), estremi inclusi
    \item \((a, b) = \{x \in \numR : a < x < b\}\), estremi esclusi
    \item \([a, +\infty) = \{x \in \numR : x \geq a\}\), semiretta
    \item \((-\infty,+\infty) = \numR\)
\end{itemize}

\pagebreak

\section{Funzioni}
Una funzione è una terna di oggetti, due insiemi e una legge.
\[\funcs{f}{A}{B}\]
\begin{itemize}
    \item \(A\) è il dominio
    \item \(B\) è il codominio
    \item \(f\) è una legge che mette in corrispondenza ogni elemento di \(A\) con \underline{uno e un solo} elemento di \(B\).
\end{itemize}

\subsection{Grafico di una funzione}
\begin{dfn}
È possibile definire il grafico di una funzione come un insieme di punti.
\[graph(f) = \{(a,b) \in A \times B :\gap b = f(a)\} \subset (A \times B)\]
\end{dfn}
\begin{eg}
\(\funcs{f}{\numR}{\numR}, \Gap f(x) = 2x\)
\begin{align*}
    f(3) = 2 \cdot 3 = 6 &\implies (3,6) \in graph(f) \\
    f(3) \neq 7 &\implies (3,7) \notin graph(f)
\end{align*}
\end{eg}

\subsection{Immagine di una funzione}
\begin{dfn}
\(f: A \rightarrow B,\Gap D \subset A,\gap f(D) \subset B\)\\
L'insieme
\[f(D) = \{f(x) : x \in D\}\]
si dice \emph{immagine} di \(D\) attraverso \(f\).\\
Nel caso \(D = A\) invece di \(f(A)\) si scrive
\[Imm(f) = f(A) = \{f(x) : x \in A \}\]
\end{dfn}

\begin{eg}
\(\funcs{f}{\numR}{\numR},\gap f(x) = x^2\)\\
Dato l'intervallo \(D = [2,3]\)\\
\(f(D) = ?\)
%drawing
\begin{align*}
x \in [2,3] \iff    &x \leq x \leq 3\\
                    &4 \leq x^2 \leq 9\\
                    &4 \leq f(x) \leq 9
\end{align*}
\begin{align*}
    & \funcs{f}{\numR}{\numR},\gap f(x) = x^2 && \funcs{f}{\numR}{\numR},\gap g(x) = -x^2\\
    & Imm(f) = [0, +\infty)                     && Imm(g) = (-\infty, 0]
\end{align*}
\end{eg}

\subsection{Iniettività e Surgettività}
\begin{dfn}
\(\funcs{f}{A}{B}\) si dice \emph{iniettiva} se per ogni coppia di \(x\) diverse appartenenti al dominio la funzione valutata in \(x_1\) ha valore diverso dalla funzione valutata in \(x_2\).
\[\forall x_1,x_2 \in A \text{ con } x_1 \neq x_2 \implies f(x_1) \neq f(x_2)\]
\end{dfn}

\begin{eg}
La funzione \(\funcs{f}{\numR}{\numR},\gap f(x) = x^2\) \underline{\textbf{non} è \emph{iniettiva}}.
\begin{align*}
    f(-2) = 4 && f(2) = 4 && -2   &\neq 2\\
              &&          && f(2) &= f(-2)
\end{align*}
\end{eg}

\(f\) è \emph{iniettiva} se \textbf{ogni} retta orizzontale \(y = k\) interseca il grafico di \(f\) \underline{al massimo} in un punto.

Si può opportunamente modificare il dominio di ogni funzione per renderla \emph{iniettiva}.

\begin{dfn}
\(\funcs{f}{A}{B}\) si dice \emph{surgettiva} se per ogni \(y\) appartenete al codominio esiste una \(x\) appartenente al dominio tale che la funzione valutata in \(x\) è uguale a \(y\).
\[\forall y \in B \gap \exists\, x \in A : f(x) = y\]
\end{dfn}

\begin{eg}
La funzione \(\funcs{f}{\numR}{\numR},\gap f(x) = x^2\) \underline{\textbf{non} è \emph{surgettiva}}.
\end{eg}

Una funzione è \emph{surgettiva} se e solo se la sua \emph{immagine} coincide con il suo \emph{codominio}. Quindi \(f\) è \emph{surgettiva} se e solo se ogni retta orizzontale \(y = k\) tracciata nel codominio interseca il grafico di \(f\) in almeno un punto.

\begin{dfn}
\(\funcs{f}{A}{B}\) si dice \emph{bigettiva} se è sia iniettiva che surgettiva.
\end{dfn}

\subsection{Funzione inversa}
La funzione inversa di \(\funcs{f}{A}{B}\), quando f è \underline{\emph{bigettiva}}, è 
\[\funcs{f^{-1}}{B}{A}\]
costruita soddisfacendo le seguenti condizioni.
\begin{align*}
    f^{-1}\left(f\left(x\right)\right) &= x \Gap \forall x \in A \\
    f\left(f^{-1}\left(x\right)\right) &= y \Gap \forall y \in B
\end{align*}

\begin{dms}
Dato che \(f\) è \emph{surgettiva}, \(\exists x \in A\) che è unico perché \(f\) è \emph{iniettiva}. Allora \(x = f^{-1}(y)\).
\end{dms}
\begin{eg}
\(\sqrt{}\) è la funzione inversa di \(\funcs{f}{[0,+\infty)}{[0,+\infty)} \gap f(x) = x^2\).
\begin{nb}
\(\sqrt{x}\) non ha senso se \(x < 0\) perché \(\sqrt{x} \geq 0 \gap \forall x\)
\[\funcs{\sqrt{}}{[0,+\infty)}{[0,+\infty)} \Gap \sqrt{4} = 2 \textbf{ non } \xcancel{\sqrt{4} = \pm 2}\]
\end{nb}
\begin{align*}
    \left(\sqrt{x}\right)^2 &= x  && \forall x \geq 0 \\
    \sqrt{x^2} &= \lvert x \rvert && \forall x \in \numR
\end{align*}
\end{eg}

\subsection{Operazioni sui grafici}
\subsubsection{Traslazioni}
\(g(x) = f(x) + a\)
\begin{itemize}
    \item \(a > 0\) traslazione verso l'alto
    \item \(a < 0\) traslazione verso il basso
\end{itemize}

\noindent\(g(x) = f(x + a)\)
\begin{itemize}
    \item \(a > 0\) traslazione verso sinistra
    \item \(a < 0\) traslazione verso destra
\end{itemize}

\subsubsection{Valore assoluto}
\(g(x) = \lvert f(x) \rvert\)

\subsection{Funzioni monotone}
\begin{dfn}
\(A, B \subset \numR, \gap \funcs{f}{A}{B}, \gap x_1, x_2 \in A, \gap x_1 < x_2\)\\
Se \(\forall x_1,x_2 \in A\):
\begin{enumerate}
    \item \(f(x_1) < f(x_2) \implies f\) è strettamente crescente
    \item \(f(x_1) \leq f(x_2) \implies f\) è debolmente crescente
    \item \(f(x_1) > f(x_2) \implies f\) è strettamente decrescente
    \item \(f(x_1) \geq f(x_2) \implies f\) è debolmente crescente
\end{enumerate}
Se valgono 1 o 3 \(f\) è strettamente \emph{monotona}. \\
Se valgono 2 o 4 \(f\) è debolmente \emph{monotona}.
\end{dfn}

Nel caso di funzione debolmente \emph{monotona} sono ammessi tratti orizzontali nel grafico.

Una funzione monotona ha un andamento sempre crescente o decrescente. \\
Se f è crescente \textbf{mantiene} l'andamento
\begin{align*}
    x_1 < x_2 &\implies f(x_1) < f(x_2) \Gap \text{strettamente crescente} \\
    x_1 < x_2 &\implies f(x_1) \leq f(x_2) \Gap \text{debolmente crescente} \\
\end{align*}
Se f è decrescente \textbf{inverte} l'andamento
\begin{align*}
    x_1 < x_2 &\implies f(x_1) > f(x_2) \Gap \text{strettamente decrescente} \\
    x_1 < x_2 &\implies f(x_1) \geq f(x_2) \Gap \text{debolmente decrescente} \\
\end{align*}

Una funzione f è strettamente crescente se e solo se
\[\frac{f(x_1) - f(x_2)}{x_1 - x_2} > 0 \gap \forall x_1 \neq x_2\]

Una funzione f è strettamente decrescente se e solo se
\[\frac{f(x_1) - f(x_2)}{x_1 - x_2} < 0 \gap \forall x_1 \neq x_2\]
\end{document}