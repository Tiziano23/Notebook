\documentclass[a4paper]{article}
\usepackage[utf8]{inputenc}
\usepackage[italian]{babel}
\usepackage[margin=1.25in]{geometry}

\usepackage{amsfonts, amssymb}
\usepackage{amsmath, amsthm}
\usepackage[thicklines]{cancel}

\usepackage{graphicx}
\usepackage{hyperref}

\hypersetup{
colorlinks=true,
linkcolor=black,
filecolor=magenta,
urlcolor=blue,
}

\newtheorem*{thm}{Teorema}

\theoremstyle{definition}
\newtheorem*{dfn}{Definizione}
\newtheorem*{oss}{Osservazione}
\newtheorem*{dms}{Dimostrazione}
\newtheorem*{eg}{Esempio}
\newtheorem*{nb}{N.B}

\newcommand{\quoted}[1]{``#1''}
\newcommand{\angled}[1]{\langle #1 \rangle}
\newcommand{\numN}{\mathbb{N}}
\newcommand{\numZ}{\mathbb{Z}}
\newcommand{\numQ}{\mathbb{Q}}
\newcommand{\numR}{\mathbb{R}}

\newcommand{\gap}{\hspace{.4cm}}
\newcommand{\ggap}{\hspace{.6cm}}
\newcommand{\Gap}{\hspace{1cm}}
\newcommand{\GAP}{\hspace{2.5cm}}

%aliases
\newcommand{\funcs}[3]{#1: #2 \rightarrow #3}

\title{Test}
\author{Tiziano Marzocchella}
\date{A.A. 2022/2023}

\begin{document}
\maketitle

% \begin{lstlisting}
% // This is pseudo code
% print("Hello World");
% for (let i = 0; i < 10; i++) {
%   print(i);
% }
% \end{lstlisting}
\begin{minted}{python}
import numpy as np
    
def incmatrix(genl1,genl2):
    m = len(genl1)
    n = len(genl2)
    M = None #to become the incidence matrix
    VT = np.zeros((n*m,1), int)  #dummy variable
    
    #compute the bitwise xor matrix
    M1 = bitxormatrix(genl1)
    M2 = np.triu(bitxormatrix(genl2),1) 

    for i in range(m-1):
        for j in range(i+1, m):
            [r,c] = np.where(M2 == M1[i,j])
            for k in range(len(r)):
                VT[(i)*n + r[k]] = 1;
                VT[(i)*n + c[k]] = 1;
                VT[(j)*n + r[k]] = 1;
                VT[(j)*n + c[k]] = 1;
                
                if M is None:
                    M = np.copy(VT)
                else:
                    M = np.concatenate((M, VT), 1)
                
                VT = np.zeros((n*m,1), int)
    
    return M
\end{minted}

\end{document}